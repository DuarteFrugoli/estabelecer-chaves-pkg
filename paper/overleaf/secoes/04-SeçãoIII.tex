\section{Implementação e Detalhes de Simulação}
\label{sec:metodos}

Esta seção apresenta detalhadamente os métodos utilizados para a implementação e simulação % avaliação 
do sistema de geração de chaves criptográficas em camada física. O ambiente de desenvolvimento de simulação é descrito, incluindo a modelagem do canal sem fio, o processo de extração de aleatoriedade a partir das observações do canal, bem como os procedimentos empregados nas etapas de reconciliação da informação e amplificação de privacidade. Adicionalmente, apresentam-se os experimentos conduzidos para análise quantitativa do desempenho do sistema sob diferentes condições de operação.

%\subsection{Descrição Geral da Simulação}

O objetivo da simulação é reproduzir, de forma matemática e computacionalmente controlada, um sistema de geração de chaves baseado nas propriedades estatísticas do canal sem fio, explorando a correlação entre as observações realizadas por dois usuários legítimos, Alice e Bob. No cenário considerado, ambos recebem sinais provenientes de uma mesma fonte transmissora, de modo que suas estimativas de canal, embora não idênticas, permanecem similares.

A simulação contempla a geração de amostras complexas de canal Rayleigh, a transmissão de símbolos através de um canal sujeito a ruído \ac{AWGN}, a aplicação de esquemas de modulação digital \ac{BPSK} e \ac{QPSK} com detecção coerente, bem como o emprego de códigos corretores de erro \ac{BCH} para a etapa de reconciliação. A amplificação de privacidade é realizada por meio do algoritmo \ac{SHA}-256, garantindo que a chave final apresente elevada entropia mesmo na presença de vazamento controlado de informação durante a comunicação pública.

Por fim, o arcabouço completo permite avaliar na Seção \ref{sec:results} como alterações nas condições do canal e nos parâmetros de comunicação afetam a taxa de concordância entre as chaves geradas e o desempenho global do sistema, por meio de métricas como a \ac{KDR} e a taxa de concordância.

%\subsection{Ambiente de Desenvolvimento}

\subsection{Linguagem e Bibliotecas Utilizadas}

As simulações foram implementadas integralmente em Python 3.11, devido à sua flexibilidade e à ampla disponibilidade de bibliotecas científicas voltadas ao processamento numérico e à análise de sinais. As principais dependências incluem \textbf{NumPy}, utilizado para geração de amostras complexas e operações matriciais; \textbf{SciPy}, empregado em funções estatísticas e rotinas auxiliares; \textbf{Matplotlib}, destinado à visualização e geração de gráficos de desempenho; \textbf{Galois}, utilizado na implementação de códigos \ac{BCH} aplicados na etapa de reconciliação da informação; \textbf{hashlib}, responsável pela implementação do algoritmo \ac{SHA}-256 empregado na amplificação de privacidade; e \textbf{tqdm}, utilizado para exibição de barras de progresso durante execuções de longa duração.

Todas as simulações foram executadas em ambiente local, em um computador equipado com processador Intel Core i7 de 11ª geração, assegurando capacidade computacional adequada para a realização dos experimentos numéricos. O gerenciamento de dependências e dos ambientes virtuais foi realizado por meio do \textit{Poetry}, uma ferramenta moderna para controle e reprodutibilidade de projetos em Python.

\subsection{Parâmetros de Simulação}

Os parâmetros controláveis do ambiente de simulação incluem o número de bits transmitidos, tipicamente \(N = 127\) correspondente ao comprimento do código \ac{BCH}, os níveis de \ac{SNR} variando no intervalo \(\mathrm{SNR} \in [0, 30]\, \mathrm{dB}\), o parâmetro de escala Rayleigh \(\sigma \in [0.5, 2.0]\), os parâmetros do código \ac{BCH} \((n,k,t)\) com \(n=127\), \(k=64\) e \(t=10\), o tipo de modulação (\ac{BPSK} ou \ac{QPSK}), e a variância do ruído \ac{AWGN} calculada a partir da \ac{SNR} desejada.

Esses parâmetros foram escolhidos de forma a representar condições realistas e cobrir uma ampla faixa de cenários físicos.

%\subsection{Estrutura do Código}

\subsection{Módulos Principais}

A implementação foi organizada em módulos independentes visando clareza e modularidade. O módulo \texttt{canal/canal.py} implementa a simulação de canal Rayleigh com ruído \ac{AWGN} e modulação \ac{BPSK}/\ac{QPSK}, enquanto \texttt{canal/modulacao.py} realiza modulação temporal realista com frequência de portadora. O módulo \texttt{codigos\_corretores/bch.py} provê codificação e decodificação \ac{BCH}, e \texttt{pilares/reconciliacao.py} implementa a reconciliação de chaves usando o protocolo \textit{code-offset}. A aplicação do \ac{SHA}-256 para obtenção da chave final é realizada em
\texttt{pilares/amplificacao\_privacidade.py}, a geração de gráficos de métricas e desempenho é feita por \texttt{visualizacao/plotkdr.py}, e operações auxiliares com sequências binárias são providas por \texttt{util/binario\_util.py}.

\subsection{Fluxo Geral de Execução}

O arcabouço da simulação segue:

\begin{enumerate}
    \item geração de sequência de bits aleatória para Alice;
    \item geração do ganho de canal \(h \sim \mathcal{CN}(0,1)\) (Rayleigh);
    \item transmissão de Alice para Bob: \(y_B = h \cdot x_A + n_B\);
    \item transmissão de Bob para Alice: \(y_A = h \cdot x_B + n_A\) (reciprocidade);
    \item demodulação e detecção dos bits recebidos;
    \item reconciliação das sequências via \ac{BCH} usando code-offset;
    \item amplificação de privacidade via \ac{SHA}-256;
    \item cálculo das métricas (\ac{KDR} pré e pós-reconciliação, \ac{KDR} pós-amplificação).
\end{enumerate}

O fluxo permite avaliar precisamente o impacto da \ac{SNR}, distância e número de amostras sobre a qualidade das chaves.

%\subsection{Modelagem do Canal}

\subsection{Canal Rayleigh}

O canal é modelado como uma variável aleatória complexa:

\begin{equation}
h \sim \mathcal{CN}(0,1),
\end{equation}

resultando em módulo Rayleigh e fase uniforme. Esse modelo captura condições de multipercurso \ac{NLoS}, típico de ambientes internos.

\subsection{Modelo de Ruído, Distância e Amostras}

Cada nó observa:

\begin{equation}
y = h x + n,
\end{equation}

onde \(x\) é o piloto e \(n \sim \mathcal{CN}(0,\sigma^2)\) representa \ac{AWGN}.

A influência da distância é incorporada por perda de percurso:

\begin{equation}
h_d = \frac{h}{d^\alpha},
\end{equation}

com \(\alpha \in [2,4]\) dependendo do ambiente.

Batches de amostras são gerados para cada estimativa, simulando múltiplas realizações do canal e permitindo análise estatística do desempenho. A correlação entre os canais observados por Alice e Bob é controlada pelo parâmetro \(\rho \in [0,1]\), sendo \(\rho = 0.9\) o valor típico utilizado neste trabalho, representando alta reciprocidade.

\subsection{Protocolo de Reconciliação Code-Offset}

A reconciliação de chaves neste trabalho é implementada utilizando o protocolo \textit{code-offset}, que combina segurança da informação com correção de erros.
O protocolo opera da seguinte forma:

\begin{enumerate}
    \item Bob gera uma sequência aleatória de \(k\) bits e a codifica usando um código \ac{BCH} \((n,k,t)\), obtendo um código \(\mathbf{c}\) de \(n\) bits;
    \item Bob calcula a síndrome pública \(\mathbf{s} = \mathbf{b}_B \oplus \mathbf{c}\), onde \(\mathbf{b}_B\) é sua sequência de bits quantizada;
    \item Bob transmite \(\mathbf{s}\) para Alice através do canal público;
    \item Alice calcula \(\mathbf{c}' = \mathbf{s} \oplus \mathbf{b}_A = \mathbf{c} \oplus \mathbf{e}\), onde \(\mathbf{e} = \mathbf{b}_A \oplus \mathbf{b}_B\) representa o vetor de erros;
    \item Alice decodifica \(\mathbf{c}'\) usando o decodificador \ac{BCH}, recuperando \(\hat{\mathbf{c}} = \mathbf{c}\) se o número de erros não exceder \(t\);
    \item Alice calcula a chave reconciliada \(\mathbf{k}_A = \mathbf{s} \oplus \hat{\mathbf{c}}\), que coincide com \(\mathbf{b}_B\) de Bob.
\end{enumerate}

O código \ac{BCH}(127,64,10) foi escolhido por oferecer bom compromisso entre taxa de código (\(R = k/n \approx 0.5\)) e capacidade de correção de até 10 erros.
Esse código é adequado para cenários de \ac{SNR} moderada, típicos em comunicações \ac{IoT} \cite{proakis_digital}.

\subsection{Métricas de Avaliação}

Para avaliar o desempenho do sistema, utiliza-se principalmente a métrica \ac{KDR}, definida como a taxa de discordância de bits entre as chaves geradas por Alice e Bob:

\begin{equation}
\text{KDR} = \frac{\text{número de bits diferentes}}{\text{número total de bits}} \times 100\%.
\end{equation}

O \ac{KDR} é calculado em três momentos distintos: o \ac{KDR} pré-reconciliação representa a discordância entre \(\mathbf{b}_A\) e \(\mathbf{b}_B\) imediatamente após a quantização, o \ac{KDR} pós-reconciliação mede a discordância após a aplicação do protocolo \textit{code-offset}, e o \ac{KDR} pós-amplificação quantifica a discordância após a aplicação do \ac{SHA}-256, devendo assumir apenas os valores 0\% ou 100\% devido à natureza determinística da função hash.

Idealmente, espera-se que o \ac{KDR} pré-reconciliação seja reduzido a 0\% após a reconciliação, desde que a capacidade de correção do código \ac{BCH} não seja excedida. Além do \ac{KDR}, avalia-se também o tempo de execução das simulações para verificar a viabilidade computacional do sistema em dispositivos com recursos limitados.

%\subsection{Experimentos Realizados}