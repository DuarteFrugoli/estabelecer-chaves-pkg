\section{Implementação e Detalhes de Simulação}
\label{sec:metodos}

Esta seção apresenta detalhadamente os métodos utilizados para a implementação, simulação e avaliação do sistema de geração de chaves em camada física.
Descrevem-se o ambiente de desenvolvimento, a modelagem do canal, o processo de extração de aleatoriedade, os procedimentos de reconciliação e amplificação de privacidade, bem como os experimentos conduzidos para análise do desempenho.

\subsection{Descrição Geral da Simulação}

O objetivo da simulação é reproduzir matematicamente e computacionalmente um sistema de geração de chaves baseado na reciprocidade do canal em comunicações sem fio. O sistema modela dois nós legítimos, Alice e Bob, trocando sinais e obtendo estimativas do canal em momentos próximos o suficiente para
garantir a reciprocidade. 

A simulação contempla a geração de amostras de canal Rayleigh complexas, a transmissão de bits através do canal com ruído \ac{AWGN}, a aplicação de modulação \ac{BPSK} ou \ac{QPSK} com detecção coerente, o uso de códigos \ac{BCH} para reconciliação, o uso de \ac{SHA}-256 para amplificação de privacidade, e o
cálculo de métricas de desempenho como \ac{KDR} e taxa de concordância.

O pipeline completo permite avaliar, de forma controlada, como alterações no cenário físico afetam a concordância das chaves e a segurança estatística do sistema.

\subsection{Ambiente de Desenvolvimento}

\subsubsection{Linguagem e Bibliotecas Utilizadas}

As simulações foram implementadas integralmente em Python 3.11 devido à sua flexibilidade e à ampla disponibilidade de bibliotecas científicas. As principais dependências incluem \textbf{NumPy} para geração de números complexos e operações matriciais, \textbf{SciPy} para funções estatísticas e transformações auxiliares, \textbf{Matplotlib} para geração de gráficos de desempenho, \textbf{Galois} para implementação de códigos \ac{BCH} utilizados na reconciliação, \textbf{hashlib} para implementação do \ac{SHA}-256 aplicado na amplificação de
privacidade, e \textbf{tqdm} para exibição de barras de progresso durante simulações longas.

Todas as simulações foram executadas em ambiente local, num computador com processador Intel Core i7 de 11ª geração, garantindo execução consistente e reprodutível. O gerenciamento de dependências foi realizado utilizando Poetry, ferramenta moderna para controle de ambientes virtuais Python.

\subsubsection{Parâmetros de Simulação}

Os parâmetros controláveis do ambiente de simulação incluem o número de bits transmitidos, tipicamente \(N = 127\) correspondente ao comprimento do código \ac{BCH}, os níveis de \ac{SNR} variando no intervalo \(\mathrm{SNR} \in [0, 30]
\, \mathrm{dB}\), o parâmetro de escala Rayleigh \(\sigma \in [0.5, 2.0]\), os parâmetros do código \ac{BCH} \((n,k,t)\) com \(n=127\), \(k=64\) e \(t=10\), o tipo de modulação (\ac{BPSK} ou \ac{QPSK}), e a variância do ruído \ac{AWGN} calculada a partir da \ac{SNR} desejada.

Esses parâmetros foram escolhidos de forma a representar condições realistas e cobrir uma ampla faixa de cenários físicos.

\subsection{Estrutura do Código}

\subsubsection{Módulos Principais}

A implementação foi organizada em módulos independentes visando clareza e modularidade. O módulo \texttt{canal/canal.py} implementa a simulação de canal Rayleigh com ruído \ac{AWGN} e modulação \ac{BPSK}/\ac{QPSK}, enquanto \texttt{canal/modulacao.py} realiza modulação temporal realista com frequência de portadora. O módulo \texttt{codigos\_corretores/bch.py} provê codificação e
decodificação \ac{BCH}, e \texttt{pilares/reconciliacao.py} implementa a reconciliação de chaves usando o protocolo \textit{code-offset}. A aplicação do
\ac{SHA}-256 para obtenção da chave final é realizada em
\texttt{pilares/amplificacao\_privacidade.py}, a geração de gráficos de métricas e desempenho é feita por \texttt{visualizacao/plotkdr.py}, e operações auxiliares
com sequências binárias são providas por \texttt{util/binario\_util.py}.

\subsubsection{Fluxo Geral de Execução}

O pipeline da simulação segue:

\begin{enumerate}
    \item geração de sequência de bits aleatória para Alice;
    \item geração do ganho de canal \(h \sim \mathcal{CN}(0,1)\) (Rayleigh);
    \item transmissão de Alice para Bob: \(y_B = h \cdot x_A + n_B\);
    \item transmissão de Bob para Alice: \(y_A = h \cdot x_B + n_A\) (reciprocidade);
    \item demodulação e detecção dos bits recebidos;
    \item reconciliação das sequências via \ac{BCH} usando code-offset;
    \item amplificação de privacidade via \ac{SHA}-256;
    \item cálculo das métricas (\ac{KDR} pré e pós-reconciliação, \ac{KDR} pós-amplificação).
\end{enumerate}

O fluxo permite avaliar precisamente o impacto da \ac{SNR}, distância e número de amostras sobre a qualidade das chaves.

\subsection{Modelagem do Canal}

\subsubsection{Canal Rayleigh}

O canal é modelado como uma variável aleatória complexa:

\begin{equation}
h \sim \mathcal{CN}(0,1),
\end{equation}

resultando em módulo Rayleigh e fase uniforme. Esse modelo captura condições de multipercurso sem linha de visada (\ac{NLoS}), típico de ambientes internos.

\subsubsection{Modelo de Ruído, Distância e Amostras}

Cada nó observa:

\begin{equation}
y = h x + n,
\end{equation}

onde \(x\) é o piloto e \(n \sim \mathcal{CN}(0,\sigma^2)\) representa \ac{AWGN}.

A influência da distância é incorporada por perda de percurso:

\begin{equation}
h_d = \frac{h}{d^\alpha},
\end{equation}

com \(\alpha \in [2,4]\) dependendo do ambiente.

Batches de amostras são gerados para cada estimativa, simulando múltiplas realizações do canal e permitindo análise estatística do desempenho. A correlação entre os canais observados por Alice e Bob é controlada pelo parâmetro \(\rho \in [0,1]\), sendo \(\rho = 0.9\) o valor típico utilizado neste trabalho, representando alta reciprocidade.

\subsection{Protocolo de Reconciliação Code-Offset}

A reconciliação de chaves neste trabalho é implementada utilizando o protocolo \textit{code-offset}, que combina segurança da informação com correção de erros.
O protocolo opera da seguinte forma:

\begin{enumerate}
    \item Bob gera uma sequência aleatória de \(k\) bits e a codifica usando um código \ac{BCH} \((n,k,t)\), obtendo um código \(\mathbf{c}\) de \(n\) bits;
    \item Bob calcula a síndrome pública \(\mathbf{s} = \mathbf{b}_B \oplus \mathbf{c}\), onde \(\mathbf{b}_B\) é sua sequência de bits quantizada;
    \item Bob transmite \(\mathbf{s}\) para Alice através do canal público;
    \item Alice calcula \(\mathbf{c}' = \mathbf{s} \oplus \mathbf{b}_A = \mathbf{c} \oplus \mathbf{e}\), onde \(\mathbf{e} = \mathbf{b}_A \oplus \mathbf{b}_B\) representa o vetor de erros;
    \item Alice decodifica \(\mathbf{c}'\) usando o decodificador \ac{BCH}, recuperando \(\hat{\mathbf{c}} = \mathbf{c}\) se o número de erros não exceder \(t\);
    \item Alice calcula a chave reconciliada \(\mathbf{k}_A = \mathbf{s} \oplus \hat{\mathbf{c}}\), que coincide com \(\mathbf{b}_B\) de Bob.
\end{enumerate}

O código \ac{BCH}(127,64,10) foi escolhido por oferecer bom compromisso entre taxa de código (\(R = k/n \approx 0.5\)) e capacidade de correção de até 10 erros.
Esse código é adequado para cenários de \ac{SNR} moderada, típicos em comunicações \ac{IoT} \cite{proakis_digital}.

\subsection{Métricas de Avaliação}

Para avaliar o desempenho do sistema, utiliza-se principalmente a métrica \ac{KDR}, definida como a taxa de discordância de bits entre as chaves geradas por Alice e Bob:

\begin{equation}
\text{KDR} = \frac{\text{número de bits diferentes}}{\text{número total de bits}} \times 100\%.
\end{equation}

O \ac{KDR} é calculado em três momentos distintos: o \textbf{KDR pré-reconciliação} representa a discordância entre \(\mathbf{b}_A\) e \(\mathbf{b}_B\) imediatamente
após a quantização, o \textbf{KDR pós-reconciliação} mede a discordância após a aplicação do protocolo \textit{code-offset}, e o \textbf{KDR pós-amplificação} quantifica a discordância após a aplicação do \ac{SHA}-256, devendo assumir apenas os valores 0\% ou 100\% devido à natureza determinística da função hash.

Idealmente, espera-se que o \ac{KDR} pré-reconciliação seja reduzido a 0\% após a reconciliação, desde que a capacidade de correção do código \ac{BCH} não seja excedida. Além do \ac{KDR}, avalia-se também o tempo de execução das simulações para verificar a viabilidade computacional do sistema em dispositivos com recursos limitados.

\subsection{Experimentos Realizados}

Foram conduzidos seis experimentos sistemáticos para avaliar o desempenho do
sistema sob diferentes condições operacionais. Cada experimento foi projetado
para isolar o impacto de um parâmetro específico sobre o \ac{KDR}.

\subsubsection{Experimento 1: Variação da \ac{SNR}}

Investiga-se o impacto da relação sinal-ruído sobre o \ac{KDR}, variando a
\ac{SNR} de -10 dB a 30 dB em 18 pontos logaritmicamente espaçados. Mantém-se
fixos \(\sigma = 1/\sqrt{2}\), modulação \ac{BPSK} e código \ac{BCH}(127,64,10).
Espera-se que o \ac{KDR} diminua monotonicamente com o aumento da \ac{SNR}.

\subsubsection{Experimento 2: Variação do Parâmetro Rayleigh}

Analisa-se o efeito do parâmetro de escala \(\sigma\) do desvanecimento Rayleigh,
testando valores \(\sigma \in \{0.5, 1/\sqrt{2}, 1.0, 2.0\}\). A \ac{SNR} é fixada
em 15 dB. Valores menores de \(\sigma\) representam canais mais severos.

\subsubsection{Experimento 3: Comparação BPSK vs QPSK}

Compara-se o desempenho dos esquemas de modulação \ac{BPSK} (1 bit/símbolo) e
\ac{QPSK} (2 bits/símbolo) em termos de \ac{KDR}. Mantém-se \ac{SNR} variável e
\(\sigma = 1/\sqrt{2}\). Espera-se que \ac{QPSK} apresente \ac{KDR} ligeiramente
superior devido à maior densidade de informação por símbolo.

\subsubsection{Experimento 4: Variação da Correlação do Canal}

Investiga-se o impacto do coeficiente de correlação \(\rho\) entre os canais de
Alice e Bob, testando valores \(\rho \in \{0.7, 0.8, 0.9, 0.95\}\). A reciprocidade
perfeita corresponde a \(\rho = 1\), enquanto valores menores simulam imperfeições
de hardware ou assimetria temporal.

\subsubsection{Experimento 5: Comparação de Códigos \ac{BCH}}

Compara-se o desempenho de diferentes códigos \ac{BCH}: (7,4,1), (15,7,2),
(127,64,10) e (255,139,15). Códigos com maior capacidade de correção (\(t\) maior)
devem apresentar melhor desempenho em cenários de alta \ac{SNR}, mas podem ser
computacionalmente mais custosos.

\subsubsection{Experimento 6: Análise de Complexidade Computacional}

Avalia-se o tempo de execução das principais operações do sistema (modulação,
codificação \ac{BCH}, decodificação, hashing) em função do tamanho do código.
O objetivo é verificar a viabilidade do sistema em dispositivos \ac{IoT} com
capacidade de processamento limitada.

Todos os experimentos foram executados com 1000 realizações de Monte Carlo no
modo completo, garantindo significância estatística dos resultados. Os dados
foram salvos em formato JSON para análise posterior, e gráficos em alta resolução
(300 DPI) foram gerados utilizando Matplotlib.

A próxima seção apresenta os resultados numéricos obtidos a partir desses
experimentos, analisando o comportamento do \ac{KDR} em diferentes cenários e
avaliando a viabilidade prática do sistema proposto.
