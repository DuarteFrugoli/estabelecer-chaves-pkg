\section{Arquitetura do Sistema Proposto}
\label{sec:modelo}

Esta seção complementa os fundamentos teóricos apresentados na Seção~II com a justificativa arquitetural específica do sistema implementado neste trabalho. A Seção~II estabeleceu os conceitos gerais de \ac{PKG} (cenário Alice-Bob-Eve, modelo de canal, correlação espacial, quantização, reconciliação e amplificação de privacidade). Aqui, apresenta-se a escolha técnica central deste trabalho: a adoção explícita do modelo baseado em \textbf{correlação espacial de canais \textit{downlink}} em detrimento do modelo tradicional de \textbf{reciprocidade temporal em sistemas TDD}, detalhando as vantagens operacionais dessa abordagem para aplicações \ac{5G} e \ac{IoT}.

\subsection{Justificativa do Modelo: Correlação Espacial vs. Reciprocidade Temporal}

A maioria dos trabalhos na literatura de \ac{PKG} explora a reciprocidade do canal em sistemas \ac{TDD} (\textit{Time Division Duplex}), nos quais Alice e Bob observam coeficientes de canal idênticos nos enlaces direto e reverso quando medidos dentro do tempo de coerência~\cite{mathur_pks}. Entretanto, esse modelo apresenta limitações práticas significativas em redes celulares modernas.

Este trabalho adota explicitamente o modelo de \textbf{correlação espacial}, no qual Alice e Bob, espacialmente próximos (distâncias típicas $d_{AB} < 0.5$~m), observam canais \textit{downlink} altamente correlacionados ($\rho \approx 0.9$) provenientes de uma mesma estação rádio base. Conforme estabelecido na Seção~II-C, a correlação espacial segue o modelo de Clarke~\cite{clarke_1968}: $\rho_{\text{espacial}}(d) = J_0(2\pi d/\lambda)$, garantindo $\rho > 0.7$ para $d < \lambda/2$. Essa escolha arquitetural justifica-se por quatro vantagens principais:

\textbf{(i) Compatibilidade com FDD e bandas assimétricas:} Sistemas \ac{5G} e \ac{NB-IoT} frequentemente operam em modo \ac{FDD} (\textit{Frequency Division Duplex}) ou utilizam bandas assimétricas (diferentes frequências para \textit{uplink} e \textit{downlink}), impossibilitando a exploração direta de reciprocidade temporal. O modelo baseado em correlação espacial aplica-se independentemente do esquema de duplexação, pois ambos os dispositivos observam o mesmo canal \textit{downlink} da estação base~\cite{zhang_pks_survey}.

\textbf{(ii) Simplicidade de implementação:} Não é necessária sincronização temporal precisa entre Alice e Bob, nem transmissão de pilotos por ambos os dispositivos. Basta a recepção de sinais \textit{downlink} de referência já disponíveis para estimação de canal em protocolos celulares padrão (\ac{5G} NR, \ac{NB-IoT}), reduzindo sobrecarga de sinalização e complexidade de hardware~\cite{3gpp_nr_coverage}.

\textbf{(iii) Escalabilidade natural:} Múltiplos pares de dispositivos próximos podem gerar chaves simultaneamente utilizando os mesmos sinais de referência da estação base, sem interferência mútua. Essa propriedade é fundamental para redes \ac{IoT} densas com milhares de dispositivos por célula~\cite{nb_iot_link_budget}.

\textbf{(iv) Segurança física intrínseca:} A descorrelação espacial segundo o modelo de Clarke garante que atacantes distantes ($d_E > \lambda/2$) observem canais estatisticamente independentes ($\rho_E \approx 0$), resultando em \ac{BER} próxima a 50\% sem necessidade de suposições adicionais sobre comportamento temporal do canal. Conforme demonstrado na Seção~II-D e validado experimentalmente no Experimento~6 (Seção~\ref{sec:results}), essa propriedade garante segurança contra espionagem passiva mesmo quando Eve intercepta toda a comunicação pública durante a reconciliação.

\subsection{Cenários de Aplicação Típicos}

O sistema proposto aplica-se a contextos de redes celulares \ac{5G} e \ac{NB-IoT} nos quais dois dispositivos legítimos (Alice e Bob) espacialmente próximos desejam estabelecer uma chave criptográfica compartilhada. Cenários práticos incluem:

\begin{itemize}
    \item \textbf{Dispositivos wearable} carregados por um mesmo usuário (relógio inteligente + smartphone), com separação típica $d_{AB} \approx 0.3$--$0.5$~m, resultando em $\rho \gtrsim 0.85$ em frequências de 2.4~GHz ($\lambda = 12.5$~cm);
    \item \textbf{Sensores IoT co-localizados} em ambientes industriais ou residenciais inteligentes, instalados a distâncias $d_{AB} < 1$~m, garantindo $\rho \gtrsim 0.7$;
    \item \textbf{Dispositivos embarcados} em um mesmo veículo (sistemas veiculares V2X), operando em frequências de 5.9~GHz com correlação espacial elevada devido à proximidade física;
    \item \textbf{Comunicação D2D assistida} por infraestrutura, na qual dispositivos próximos exploram sinais de âncora da estação base como fonte comum de aleatoriedade.
\end{itemize}

Para garantir correlação espacial adequada ($\rho > 0.7$), adota-se neste trabalho o valor típico $\rho = 0.9$, representativo de dispositivos com separação $d_{AB} \approx 0.1\lambda$. Em frequências de 900~MHz ($\lambda \approx 33$~cm), isso corresponde a $d_{AB} \approx 3.3$~cm; em 2.4~GHz ($\lambda = 12.5$~cm), a $d_{AB} \approx 1.25$~cm. Esses valores são plenamente alcançáveis nos cenários descritos anteriormente.

A segurança contra espionagem passiva é garantida pela separação mínima $d_E \geq 20$~cm entre usuários legítimos e atacante Eve, distância na qual a correlação espacial $\rho_{\text{espacial}}(d_E) < 0.02$ torna-se desprezível (Experimento~6, Seção~\ref{sec:results}), assegurando \ac{BER} de Eve próxima a 50\% e impossibilitando inferência da chave final mesmo com conhecimento completo da informação pública trocada durante a reconciliação.

\subsection{Extensão para Cenários Device-to-Device}

Embora este trabalho foque no cenário de dispositivos recebendo sinais \textit{downlink} de uma estação base comum, o modelo de correlação espacial pode ser estendido para cenários \ac{D2D} (\textit{Device-to-Device}). Nessa configuração, dispositivos próximos observam canais espacialmente correlacionados em relação a uma fonte transmissora comum, que pode ser outro dispositivo \ac{D2D} atuando como transmissor de referência, a própria estação base operando como âncora de segurança, ou múltiplas estações base em cenários de cooperação celular. A validação experimental dessa extensão e a análise de seus requisitos de sincronização e coordenação constituem direções promissoras para trabalhos futuros (Seção~VI).
