\section{Modelo de Sistema}
\label{sec:modelo}

Esta seção descreve o cenário físico e o modelo de comunicação adotado para a geração de chaves criptográficas em camada física. O sistema considera dispositivos \ac{IoT} conectados a uma estação rádio base comum, explorando a correlação espacial entre os canais \textit{downlink} para estabelecimento de chaves secretas compartilhadas, enquanto garante segurança contra atacantes passivos por meio da descorrelação espacial natural.

\subsection{Cenário de Operação}

O sistema proposto opera no contexto de redes celulares de quinta geração (\ac{5G}) e tecnologias \ac{IoT}, em que múltiplos dispositivos estabelecem comunicação com uma estação rádio base (gNodeB). Considera-se especificamente o cenário em que dois dispositivos legítimos, denominados Alice e Bob, encontram-se espacialmente próximos (tipicamente a distâncias inferiores a 1~metro) e desejam estabelecer uma chave criptográfica compartilhada para proteger suas comunicações futuras.

Ambos os dispositivos recebem sinais de referência (\textit{pilots}) transmitidos pela estação base em canal \textit{downlink}, estimam seus respectivos coeficientes de canal, e utilizam essas estimativas como fonte de aleatoriedade comum para geração da chave. A proximidade espacial entre Alice e Bob garante que seus canais apresentem correlação estatística significativa, enquanto um atacante passivo (Eve), localizado a uma distância superior, observa um canal estatisticamente independente.

\subsection{Modelo Matemático}

A estação base transmite um sinal de referência $x$ conhecido por todos os dispositivos. Os sinais recebidos por Alice e Bob são modelados como
\begin{equation}
y_A = h_A \, x + n_A, \quad y_B = h_B \, x + n_B,
\end{equation}
em que $h_A, h_B \in \mathbb{C}$ denotam os coeficientes complexos de canal entre a estação base e cada dispositivo, representando desvanecimento Rayleigh plano, e $n_A, n_B \sim \mathcal{CN}(0, \sigma^2_n)$ correspondem ao ruído \ac{AWGN} independente em cada receptor.

Os coeficientes de canal são modelados como variáveis aleatórias complexas gaussianas $h_A, h_B \sim \mathcal{CN}(0,1)$, cujos módulos seguem distribuição Rayleigh e cujas fases são uniformemente distribuídas em $[0, 2\pi]$. Esse modelo captura condições de propagação multipercurso sem linha de visada direta (\ac{NLoS}), típicas de ambientes internos e urbanos densos.

\subsection{Correlação Espacial entre Alice e Bob}

Devido à proximidade física entre Alice e Bob, os canais $h_A$ e $h_B$ apresentam correlação estatística significativa. Essa correlação é modelada por
\begin{equation}
h_B = \rho \, h_A + \sqrt{1 - \rho^2} \, h_{\text{ind}},
\end{equation}
em que $\rho \in [0,1]$ é o coeficiente de correlação espacial e $h_{\text{ind}} \sim \mathcal{CN}(0,1)$ representa uma componente independente de $h_A$.

Conforme apresentado na Seção~II, o valor de $\rho$ é determinado pela separação física $d$ entre os dispositivos segundo o modelo de Clarke~\cite{clarke_1968}:
\begin{equation}
\rho_{\text{espacial}}(d) = J_0\left( \frac{2\pi d}{\lambda} \right),
\end{equation}
em que $J_0(\cdot)$ denota a função de Bessel de primeira espécie e ordem zero, e $\lambda$ é o comprimento de onda da portadora. Para dispositivos próximos, com separação $d < \lambda/2$, tem-se tipicamente $\rho > 0.7$, garantindo elevada similaridade entre as observações de canal realizadas por Alice e Bob.

Neste trabalho, adota-se $\rho = 0.9$ como valor típico, correspondente a uma separação espacial de aproximadamente $d \approx 0.1\lambda$ em cenários de alta correlação. Esse valor representa situações práticas como dispositivos \textit{wearable} carregados por um mesmo usuário, sensores \ac{IoT} instalados próximos, ou dispositivos embarcados em um mesmo veículo.

\subsection{Segurança contra Espionagem Passiva}

A segurança do sistema baseia-se na descorrelação espacial natural entre os canais observados pelos usuários legítimos (Alice e Bob) e por um atacante passivo (Eve). Assume-se que Eve possui capacidade ilimitada de escuta do canal público (podendo interceptar toda a informação trocada durante a reconciliação), mas não pode alterar ou injetar mensagens (\textit{eavesdropper passivo}).

Eve observa o sinal transmitido pela estação base através de um canal independente:
\begin{equation}
y_E = h_E \, x + n_E,
\end{equation}
em que $h_E \sim \mathcal{CN}(0,1)$ representa o coeficiente de canal entre a estação base e Eve.

Devido à separação espacial entre Eve e os usuários legítimos, o canal $h_E$ apresenta correlação desprezível com $h_A$ e $h_B$. Especificamente, para distâncias $d_E > \lambda/2$ (aproximadamente 6.25~cm em frequências de 2.4~GHz), a correlação espacial $\rho_{\text{espacial}}(d_E) \approx 0$, garantindo que as observações de Eve sejam estatisticamente independentes das observações de Alice e Bob.

Nesse regime, mesmo que Eve conheça toda a informação pública trocada durante a reconciliação (síndrome transmitida via \textit{code-offset}), sua capacidade de inferir a chave final é equivalente a uma tentativa de adivinhação aleatória, caracterizada por uma \ac{BER} próxima a 50\%. Esse comportamento é validado experimentalmente no Experimento~6 apresentado na Seção~IV.

\subsection{Justificativa do Modelo: Correlação Espacial vs. Temporal}

Este trabalho adota o modelo de correlação espacial (dispositivos próximos observando canais \textit{downlink} correlacionados) em detrimento do modelo de reciprocidade temporal tradicional (exploração da reciprocidade do canal em sistemas TDD). A escolha justifica-se pelos seguintes fatores:

\begin{itemize}
    \item \textbf{Compatibilidade com 5G/IoT}: Sistemas \ac{5G} e \ac{NB-IoT} frequentemente operam em modo FDD ou utilizam bandas assimétricas, impossibilitando exploração direta de reciprocidade temporal. O modelo espacial aplica-se independentemente da duplex;
    
    \item \textbf{Simplicidade de implementação}: Não requer sincronização temporal precisa nem transmissão de pilotos por Alice e Bob, apenas recepção de sinais \textit{downlink} já disponíveis para estimação de canal em protocolos celulares padrão;
    
    \item \textbf{Escalabilidade}: Múltiplos pares de dispositivos próximos podem gerar chaves simultaneamente sem interferência, utilizando os mesmos sinais de referência da estação base;
    
    \item \textbf{Segurança intrínseca}: A descorrelação espacial garante segurança física contra atacantes distantes sem necessidade de suposições adicionais sobre o comportamento temporal do canal.
\end{itemize}

Adicionalmente, o modelo pode ser estendido para cenários \textit{Device-to-Device} (\ac{D2D}), nos quais dispositivos próximos observam canais correlacionados espacialmente em relação a uma fonte transmissora comum (que pode ser outro dispositivo \ac{D2D} ou a própria estação base operando como âncora).
