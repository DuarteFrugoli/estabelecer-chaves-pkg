\section{Conclusões}\label{sec:conclusao}

Este trabalho apresentou um sistema completo de geração de chaves criptográficas em camada física para dispositivos \ac{IoT}, explorando a correlação espacial entre canais \textit{downlink} de dispositivos próximos conectados a uma mesma estação rádio base. A abordagem proposta foi validada experimentalmente por meio de sete experimentos sistemáticos, cobrindo aspectos de desempenho, segurança, complexidade computacional e aplicabilidade prática em múltiplos cenários de operação.

\subsection{Principais Contribuições}

Os resultados experimentais demonstram a viabilidade técnica e operacional do sistema proposto. Os principais achados incluem:

\begin{itemize}
    \item \textbf{Requisitos operacionais alcançáveis}: O sistema requer \ac{SNR} mínima de $13$--$15$~dB para operação confiável (KDR nula), valor plenamente alcançável em enlaces de comunicação típicos de redes celulares \ac{5G} e \ac{NB-IoT};
    
    \item \textbf{Baixa complexidade computacional}: O tempo de processamento de $0.489$~ms para reconciliação via código \ac{BCH}(127,64,10) e amplificação de privacidade via \ac{SHA}-256 viabiliza implementação em dispositivos de baixo custo operando em software, sem necessidade de hardware dedicado (FPGA, USRP), com capacidade teórica superior a $2000$ operações por segundo;
    
    \item \textbf{Ampla aplicabilidade em cenários IoT}: O sistema foi validado em cinco perfis representativos de dispositivos \ac{IoT} — sensor estático, \textit{wearable}/pessoa andando, veículo urbano ($60$~km/h), drone ($40$~km/h) e dispositivo NB-IoT — demonstrando operação bem-sucedida em condições diversas de mobilidade e frequência. Notavelmente, o cenário de veículo urbano funciona adequadamente mesmo com correlação temporal relativamente baixa ($\rho = 0.16$), confirmando que o erro de estimação de canal, quando controlado ($\leq 30\%$), é mais crítico para a viabilidade do sistema do que a correlação temporal propriamente dita;
    
    \item \textbf{Segurança física intrínseca}: A análise de segurança contra espionagem passiva confirma que distâncias laterais superiores a $20$~cm (aproximadamente $3.2\lambda/2$ em frequências de $2.4$~GHz) garantem descorrelação espacial suficiente para que um atacante passivo (Eve) observe \ac{BER} próxima a $50\%$, equivalente a uma tentativa de adivinhação aleatória. Os resultados validam experimentalmente o modelo teórico de Clarke~\cite{clarke_1968} e demonstram que a descorrelação espacial é suficiente para proteção, sendo a sincronização temporal do atacante irrelevante;
    
    \item \textbf{Dispensa de guard-band}: Uma contribuição original deste trabalho é a demonstração experimental de que técnicas de limiarização adaptativa (\textit{guard-band}) não são necessárias para garantir a segurança do sistema. O sistema opera de forma naturalmente segura sem \textit{guard-band} (GB $= 0$), apresentando \ac{BER} de Eve próxima a $50\%$. Valores elevados de \textit{guard-band} (GB $> 0.5\sigma$) são contraproducentes, reduzindo drasticamente a taxa efetiva de geração de chaves (descarte superior a $30\%$ dos bits) sem ganho de segurança, e em alguns casos introduzindo viés estatístico que pode degradar ligeiramente a proteção contra espionagem.
\end{itemize}

\subsection{Diferencial em Relação ao Estado da Arte}

O sistema proposto distingue-se de trabalhos anteriores em aspectos fundamentais. Enquanto abordagens tradicionais como Yuan et al.~\cite{yuan_fast_2013} requerem plataformas de hardware especializadas (USRP, FPGA) para extração de características do canal WiFi CSI, este trabalho implementa o sistema completo em software Python, reduzindo significativamente os requisitos de custo e complexidade para aplicações práticas de \ac{IoT}.

A validação em múltiplos perfis de dispositivos \ac{IoT} (sensor estático, \textit{wearable}, veículo, drone, NB-IoT), cada um caracterizado por parâmetros físicos realistas de mobilidade, frequência e condições de canal, representa uma extensão importante em relação a trabalhos que avaliam o desempenho apenas em cenários idealizados ou estáticos. A demonstração de que o sistema opera adequadamente mesmo em cenários de alta mobilidade (veículo a $60$~km/h, correlação temporal $\rho = 0.16$) sugere que a robustez da estimação de canal é mais relevante que a correlação temporal propriamente dita, um insight que pode guiar o desenvolvimento de futuros sistemas práticos.

Finalmente, a análise sistemática do impacto do \textit{guard-band} (Experimento~7) representa uma contribuição metodológica relevante. Este trabalho demonstra, pela primeira vez de forma experimental abrangente, que a utilização de zonas mortas na quantização não apenas é desnecessária quando a descorrelação espacial é adequadamente explorada, mas pode ser prejudicial à eficiência do sistema sem oferecer ganhos de segurança mensuráveis.

\subsection{Limitações Identificadas}

Algumas limitações do sistema foram identificadas durante os experimentos:

\begin{itemize}
    \item \textbf{Requisito mínimo de SNR}: Embora alcançável em aplicações práticas, o \ac{SNR} mínimo de $13$--$15$~dB pode ser desafiador em ambientes extremamente ruidosos ou em dispositivos operando com potência muito limitada. Cenários com \ac{SNR} inferior a $10$~dB apresentam \ac{KDR} superior a $2\%$, comprometendo a eficiência da reconciliação;
    
    \item \textbf{Sensibilidade ao erro de estimação}: O desempenho do sistema é fortemente dependente da qualidade da estimação de canal. Erros de estimação superiores a $30\%$ podem degradar significativamente o \ac{KDR}, especialmente em cenários de alta mobilidade. Estratégias robustas de estimação de canal são essenciais para extensão do sistema a condições adversas;
    
    \item \textbf{Validação em simulação}: Todos os experimentos foram conduzidos por meio de simulações numéricas de Monte Carlo. Embora os modelos adotados (desvanecimento Rayleigh, correlação espacial de Clarke, ruído \ac{AWGN}) sejam amplamente aceitos e validados na literatura, uma validação experimental em ambiente real com dispositivos físicos é necessária para confirmar os resultados obtidos e identificar eventuais efeitos não capturados pela simulação.
\end{itemize}

\subsection{Trabalhos Futuros}

Diversas direções de pesquisa podem ser exploradas como extensão deste trabalho:

\begin{itemize}
    \item \textbf{Implementação em plataforma real}: Validação experimental do sistema em dispositivos \ac{IoT} comerciais (ESP32, Raspberry Pi, módulos NB-IoT) operando em redes celulares reais (\ac{5G}, \ac{NB-IoT}), visando confirmar os resultados de simulação e identificar desafios práticos de implementação;
    
    \item \textbf{Expansão para novos perfis de dispositivos}: Investigação de cenários adicionais não cobertos neste trabalho, incluindo dispositivos aéreos de alta velocidade (aeronaves não tripuladas, UAVs), aplicações submarinas, e dispositivos operando em bandas milimétricas (mmWave) típicas de redes \ac{5G};
    
    \item \textbf{Análise de segurança contra ataques ativos}: Extensão da análise de segurança para considerar atacantes ativos capazes de manipular ou injetar mensagens no canal público durante a reconciliação, avaliando a necessidade de mecanismos adicionais de autenticação;
    
    \item \textbf{Otimização adaptativa de parâmetros}: Desenvolvimento de algoritmos para ajuste dinâmico de parâmetros do sistema (código \ac{BCH}, modulação, \textit{guard-band}) em função das condições instantâneas do canal, visando otimizar o compromisso entre taxa de geração de chaves, robustez e eficiência energética;
    
    \item \textbf{Extensão para comunicação Device-to-Device (D2D)}: Investigação da aplicabilidade do modelo de correlação espacial em cenários de comunicação direta entre dispositivos (D2D), explorando a correlação de canais em relação a fontes transmissoras comuns ou a múltiplas estações base em cenários de cooperação;
    
    \item \textbf{Integração com protocolos criptográficos}: Desenvolvimento de protocolos completos de estabelecimento de sessão segura integrando o sistema de geração de chaves em camada física com primitivas criptográficas padrão (AES, autenticação via HMAC), avaliando o impacto na latência total e na segurança do sistema end-to-end.
\end{itemize}

\subsection{Considerações Finais}

Este trabalho demonstra que a geração de chaves criptográficas em camada física, explorando a correlação espacial entre canais de dispositivos \ac{IoT} próximos, constitui uma abordagem viável e promissora para estabelecimento de segurança em redes sem fio de próxima geração. A combinação de baixa complexidade computacional, ampla aplicabilidade em múltiplos cenários de operação, e segurança física intrínseca sem necessidade de técnicas complexas de limiarização, posiciona o sistema proposto como uma solução prática e acessível para proteção de comunicações em ambientes \ac{IoT} de larga escala.

Os resultados obtidos confirmam que, quando adequadamente projetado, um sistema de geração de chaves em camada física pode operar de forma eficiente mesmo em condições desafiadoras de mobilidade e variabilidade de canal, desde que a qualidade da estimação de canal seja mantida em níveis aceitáveis. A validação experimental em plataformas reais permanece como próximo passo essencial para consolidação da tecnologia e sua eventual adoção em aplicações comerciais de \ac{IoT} e comunicações \ac{5G}/\ac{6G}.