\section{Introdução}

A evolução das comunicações móveis, impulsionada pelo desenvolvimento da \ac{5G} e pelas pesquisas da \ac{6G}, tem intensificado a demanda por mecanismos de segurança mais eficientes e adequados às características do ambiente sem fio. O crescimento do número de dispositivos conectados, aliado ao aumento da troca de informações sensíveis, torna a garantia da segurança dos dados transmitidos um desafio central para pesquisadores e engenheiros de telecomunicações~\cite{cirani_iot}.

Uma comunicação segura deve atender a cinco pilares fundamentais da segurança da informação~\cite{stallings_crypto}. A confidencialidade visa garantir que apenas as partes autorizadas tenham acesso ao conteúdo transmitido, prevenindo a interceptação e o vazamento de dados. A integridade assegura que a informação não seja modificada de forma indevida durante o armazenamento ou a transmissão, preservando seu conteúdo original. A autenticidade tem como objetivo confirmar que as entidades envolvidas na comunicação são legítimas, mitigando ataques de falsificação e impersonificação. A irretratabilidade, ou não repúdio, busca impedir que uma das partes envolvidas negue posteriormente sua participação em uma comunicação ou transação. Por fim, a disponibilidade garante que dados e serviços permaneçam acessíveis sempre que necessários, evitando interrupções causadas por falhas ou ataques.

Dentre esses pilares, a confidencialidade desempenha um papel central na proteção das informações trafegadas. O principal mecanismo técnico para garantir a confidencialidade dos dados é a criptografia. Algoritmos criptográficos têm como objetivo transformar a informação original (texto claro) em uma versão cifrada (texto cifrado), de modo que apenas entidades autorizadas possam recuperar o conteúdo original~\cite{stallings_crypto}. Para realizar essa transformação, são necessárias chaves criptográficas, que atuam como parâmetros secretos utilizados tanto no processo de cifragem quanto no de decifragem. No caso da criptografia simétrica, como o \ac{AES}, ambas as partes envolvidas na comunicação devem compartilhar a mesma chave secreta, sendo fundamental que essa chave não seja transmitida pelo canal em forma clara, pois sua interceptação comprometeria toda a segurança do sistema. Surge, assim, o problema do estabelecimento de um canal seguro para a troca inicial de chaves criptográficas. Os métodos convencionais para tratar esse problema incluem o protocolo Diffie–Hellman~\cite{diffie_hellman} e infraestruturas de chave pública baseadas em algoritmos criptográficos assimétricos, como o \ac{RSA}. No entanto, esses métodos apresentam desvantagens relevantes em cenários de redes móveis e \ac{IoT}, pois envolvem elevado custo computacional, limitando sua aplicação em dispositivos com recursos restritos, além de apresentarem vulnerabilidades crescentes diante do avanço da computação quântica~\cite{mosca_quantum}.

Esse cenário reforça a necessidade de mecanismos complementares para o estabelecimento de segredos, especialmente em dispositivos móveis e em ambientes com restrições de capacidade computacional e energética. Entre as alternativas emergentes, destaca-se a \ac{PLS}, uma abordagem que explora propriedades físicas intrínsecas do canal de comunicação sem fio — tais como a reciprocidade do canal, variações temporais e ruído térmico — para permitir o estabelecimento de chaves simétricas entre dispositivos comunicantes de forma confidencial, sem a necessidade de transmissão explícita de chaves. Dessa forma, reduz-se o custo computacional e aumenta-se a resiliência do sistema a ataques~\cite{bloch_wireless_security, zhou_pls_survey}.

Nesse contexto, a \ac{PKG} emerge como uma técnica promissora que permite a dois dispositivos legítimos estabelecerem uma chave criptográfica comum a partir da exploração das características aleatórias e recíprocas do canal de comunicação~\cite{mathur_pks, zhang_pks_survey}. Diferentemente dos métodos tradicionais, que dependem de infraestruturas de chave pública ou da pré-distribuição de segredos, a \ac{PKG} utiliza as flutuações naturais do canal sem fio para gerar bits de entropia compartilhados entre transmissor e receptor, desde que ambos realizem medições do canal em um curto intervalo de tempo, respeitando o tempo de coerência do canal, e que o ambiente apresente variações suficientes para garantir a geração de entropia.

Essa abordagem é particularmente relevante para redes \ac{5G}/\ac{6G} e sistemas \ac{IoT}, nos quais escalabilidade, baixo consumo energético e maior resiliência a ameaças quânticas constituem requisitos críticos~\cite{zeng_pkg_challenges}. Em redes densas, com milhares de dispositivos conectados simultaneamente, mecanismos que não dependem de infraestrutura centralizada de gerenciamento de chaves tornam-se essenciais para garantir a viabilidade operacional. Além disso, dispositivos \ac{IoT} frequentemente operam com recursos energéticos limitados, o que inviabiliza o uso intensivo de algoritmos criptográficos computacionalmente onerosos. A \ac{PKG}, ao explorar o canal físico já empregado para a comunicação, oferece uma solução energeticamente eficiente, uma vez que as estimações do canal são realizadas durante o próprio processo comunicacional, sem a necessidade de transmissões adicionais ou de processamento criptográfico complexo, reduzindo significativamente o consumo de energia quando comparada a protocolos tradicionais de estabelecimento de chaves~\cite{zeng_pkg_challenges}. Por fim, diante do avanço da computação quântica, algoritmos clássicos como \ac{RSA} e Diffie–Hellman tornam-se progressivamente vulneráveis, enquanto a \ac{PKG}, por não depender de problemas matemáticos computacionalmente complexos, apresenta maior resiliência a esse tipo de ameaça~\cite{mosca_quantum}.

O presente trabalho apresenta a implementação e a análise de um sistema completo de comunicação móvel entre dois dispositivos \ac{IoT}, voltado à geração de chaves criptográficas por meio de \ac{PKG}. A proposta investiga a viabilidade prática e o desempenho do sistema sob diferentes condições de canal e parâmetros de comunicação, tais como a \ac{SNR}, os parâmetros de desvanecimento ($\sigma$), esquemas de modulação digital (\ac{BPSK} e \ac{QPSK}) e um protocolo de reconciliação do tipo \textit{code-offset}, com suporte de códigos corretores de erro \ac{BCH}~\cite{proakis_digital}. Os resultados experimentais demonstram, de forma geral, um estabelecimento eficiente de chaves criptográficas sob diversas condições de operação, especialmente quando os dispositivos experimentam canais com elevada correlação.

O restante deste artigo está organizado da seguinte forma: a Seção II apresenta a fundamentação teórica sobre \ac{PKG}, seus principais componentes e o modelo de sistema; a Seção III descreve os detalhes de implementação e simulação; a Seção IV apresenta os resultados numéricos; e a Seção V apresenta as conclusões.