\section{Introdução}

A evolução das comunicações móveis, impulsionada pelo desenvolvimento das redes \ac{5G} e pelas pesquisas em \ac{6G}, tem intensificado a demanda por mecanismos de segurança mais eficientes e adaptados ao ambiente sem fio. Com o crescimento do número de dispositivos conectados e da troca de informações sensíveis, garantir a segurança dos dados transmitidos tornou-se um desafio central para pesquisadores e engenheiros de telecomunicações.

Uma comunicação segura deve atender a cinco pilares fundamentais da Segurança da Informação. A confidencialidade garante que apenas as partes autorizadas possam ter acesso ao conteúdo transmitido, impedindo a interceptação e o vazamento de dados. A integridade assegura que a informação não seja alterada de forma indevida durante o armazenamento ou transmissão, preservando seu conteúdo original. A autenticidade assegura que as entidades envolvidas na comunicação são genuínas, evitando ataques de falsificação e impersonificação. A irretratabilidade (ou não-repúdio) impede que uma das partes envolvidas neguem posteriormente ter participado de uma comunicação ou transação. Por fim, a disponibilidade garante que os dados e serviços estejam acessíveis sempre que necessário, evitando interrupções causadas por falhas ou ataques.

Dentre esses pilares, a confidencialidade desempenha um papel central na proteção das informações trafegadas. Sua implementação tradicional é realizada por algoritmos criptográficos, como \ac{AES} e \ac{RSA}, que dependem diretamente do estabelecimento seguro de chaves criptográficas simétricas ou assimétricas. No entanto, os métodos convencionais de troca de chaves — como Diffie–Hellman e infraestruturas de chave pública — envolvem elevado custo computacional e apresentam vulnerabilidades crescentes diante do avanço da computação quântica.

Esse cenário reforça a necessidade de mecanismos complementares para o estabelecimento de segredos, especialmente em dispositivos móveis e ambientes com capacidade computacional e energética limitadas. Entre as alternativas emergentes, destaca-se a \ac{PLS}, abordagem que explora propriedades físicas intrínsecas do canal de comunicação sem fio — como reciprocidade do canal, variações temporais e ruído térmico — para estabelecer chaves simétricas em ambos os dispositivos sem depender de transmissão explícita de chaves, reduzindo o custo computacional e aumentando a resiliência contra ataques.

Dentro desse contexto, o \ac{PKG} emerge como uma técnica promissora que permite a dois dispositivos legítimos estabelecerem uma chave criptográfica comum explorando as características aleatórias e recíprocas do canal de comunicação. Diferentemente dos métodos tradicionais que exigem infraestrutura de chave pública ou pré-distribuição de segredos, o \ac{PKG} utiliza as flutuações naturais do canal sem fio para gerar bits de entropia compartilhados entre transmissor e receptor, desde que ambos realizem medições do canal em um curto intervalo de tempo, respeitando o tempo de coerência do canal, e que o ambiente apresente variações suficientes para gerar entropia. Esta abordagem é particularmente relevante para redes \ac{5G}/\ac{6G} e sistemas \ac{IoT}, onde a escalabilidade, o baixo consumo energético e a maior resiliência contra ameaças quânticas são requisitos críticos. Em redes densas com milhares de dispositivos conectados simultaneamente, mecanismos que não dependem de infraestrutura centralizada de gerenciamento de chaves tornam-se essenciais para garantir a viabilidade operacional. Adicionalmente, dispositivos \ac{IoT} frequentemente operam com baterias limitadas, tornando inviável o uso intensivo de algoritmos criptográficos pesados. O \ac{PKG}, ao explorar o canal físico já utilizado para comunicação, oferece uma solução energeticamente eficiente. Por fim, com o avanço dos computadores quânticos, algoritmos tradicionais como RSA e Diffie–Hellman tornam-se vulneráveis, enquanto o \ac{PKG}, ao não depender de problemas matemáticos computacionalmente complexos, apresenta maior resiliência a esses ataques.

O presente trabalho apresenta a implementação e análise de um sistema completo de comunicação móvel que objetiva \ac{PKG}, investigando sua viabilidade prática e desempenho sob diferentes condições de canal e parâmetros de comunicação, tais como relação sinal-ruído (SNR), parâmetros de desvanecimento (sigma), modulação digital (BPSK/QPSK), e protocolo de reconciliação code-offset com suporte de códigos corretores de erro BCH. Os resultados experimentais demonstram, de forma geral, um bom estabelecimento de chaves criptográficas sob diversas condições de operação.

O restante deste artigo está organizado da seguinte forma: a Seção II apresenta a fundamentação teórica sobre \ac{PKG} e seus componentes principais; a Seção III descreve o modelo de sistema proposto; a Seção IV detalha a implementação e os parâmetros de simulação; e a Seção V apresenta os resultados obtidos e conclusões.
