\section{Introdução}

A evolução das comunicações móveis, impulsionada pelo avanço da \ac{5G} e pelas pesquisas voltadas à \ac{6G}, tem intensificado a demanda por mecanismos de segurança mais eficientes e compatíveis com as particularidades do ambiente sem fio. O crescimento expressivo no número de dispositivos conectados, aliado ao aumento na troca de informações sensíveis, torna a proteção dos dados transmitidos um dos desafios centrais para pesquisadores e engenheiros de telecomunicações~\cite{cirani_iot}.

Uma comunicação segura deve contemplar cinco pilares fundamentais da segurança da informação~\cite{stallings_crypto}. A confidencialidade busca garantir que apenas entidades autorizadas tenham acesso ao conteúdo transmitido, prevenindo interceptações e vazamentos. A integridade assegura que a informação não seja alterada de forma indevida durante o armazenamento ou a transmissão, preservando seu conteúdo original. A autenticidade tem como objetivo confirmar a legitimidade das entidades participantes, mitigando ataques de falsificação e impersonificação. A irretratabilidade, ou não repúdio, impede que uma das partes negue posteriormente sua participação em uma comunicação ou transação. Por fim, a disponibilidade garante que dados e serviços permaneçam acessíveis sempre que necessário, evitando interrupções causadas por falhas ou ataques.

Dentre esses pilares, a confidencialidade desempenha um papel central na proteção das informações trafegadas. O principal mecanismo técnico para garanti-la é a criptografia, cuja finalidade consiste em transformar a informação original (texto claro) em uma versão cifrada (texto criptografado), de modo que apenas entidades autorizadas possam recuperar o conteúdo original~\cite{stallings_crypto}. Para isso, são necessárias chaves criptográficas, que atuam como parâmetros secretos no processo de cifragem e decifragem. No caso da criptografia simétrica, como no algoritmo \ac{AES}, transmissor e receptor devem compartilhar a mesma chave secreta. Assim, torna-se fundamental que essa chave não seja transmitida em forma clara pelo canal, pois sua interceptação comprometeria toda a segurança do sistema.
Surge, portanto, o desafio do estabelecimento inicial de um segredo compartilhado entre as partes comunicantes. Métodos convencionais para tratar esse problema incluem o protocolo Diffie–Hellman~\cite{diffie_hellman} e infraestruturas de chave pública baseadas em algoritmos assimétricos, como o \ac{RSA}. Entretanto, tais soluções apresentam limitações em redes móveis e em cenários de \ac{IoT}, uma vez que envolvem elevado custo computacional, dificultando sua aplicação em dispositivos com recursos restritos. Além disso, esses métodos tornam-se progressivamente vulneráveis diante dos avanços da computação quântica~\cite{mosca_quantum}.

Esse contexto reforça a necessidade de mecanismos complementares para o estabelecimento de segredos, especialmente em dispositivos com restrições computacionais e energéticas. Entre as alternativas emergentes, destaca-se a \ac{PLS}, uma abordagem que explora propriedades físicas intrínsecas do canal de comunicação sem fio — como reciprocidade, variações temporais e ruído térmico — para possibilitar o estabelecimento confidencial de chaves simétricas sem a necessidade de transmissão explícita dessas chaves. Dessa forma, reduz-se o custo computacional e aumenta-se a resiliência do sistema a ataques~\cite{bloch_wireless_security, zhou_pls_survey}.

Nesse cenário, a \ac{PKG} emerge como uma técnica promissora, permitindo que dois dispositivos legítimos estabeleçam uma chave criptográfica comum a partir das características aleatórias e recíprocas do canal~\cite{mathur_pks, zhang_pks_survey}. Diferentemente de métodos tradicionais, que dependem de infraestruturas de chave pública ou da pré-distribuição de segredos, a \ac{PKG} utiliza as flutuações naturais do canal para gerar bits de entropia compartilhados entre transmissor e receptor. Para isso, ambos devem realizar medições do canal dentro de um curto intervalo de tempo, respeitando o tempo de coerência, e o ambiente deve apresentar variações suficientes para garantir entropia adequada.

Essa abordagem é particularmente relevante em redes \ac{5G}/\ac{6G} e em sistemas \ac{IoT}, nos quais escalabilidade, baixo consumo energético e maior resiliência a ameaças quânticas constituem requisitos críticos~\cite{zeng_pkg_challenges}. Em redes densas, com milhares de dispositivos conectados simultaneamente, mecanismos independentes de infraestrutura centralizada de gerenciamento de chaves tornam-se essenciais para garantir viabilidade operacional. Além disso, dispositivos \ac{IoT} frequentemente operam com limitações energéticas, o que inviabiliza o uso intensivo de algoritmos criptográficos computacionalmente onerosos. Nesse sentido, a \ac{PKG}, ao explorar correlação espacial de canais observados por dispositivos próximos conectados à mesma estação base, oferece solução energeticamente eficiente sem necessidade de infraestrutura adicional dedicada ao gerenciamento de chaves~\cite{zeng_pkg_challenges}.

O modelo de sistema adotado neste trabalho considera dispositivos \ac{IoT} conectados à mesma estação base, explorando correlação espacial entre os canais \textit{downlink} observados por usuários legítimos em proximidade física. Essa arquitetura é particularmente relevante para redes celulares 5G e NB-IoT, nas quais múltiplos dispositivos compartilham a mesma infraestrutura de acesso e podem estabelecer chaves criptográficas sem necessidade de comunicação direta entre si ou de canais adicionais dedicados à distribuição de chaves. A segurança do sistema fundamenta-se na descorrelação espacial do canal de um atacante passivo (\textit{eavesdropper}) localizado a distância suficiente dos usuários legítimos, condição que será validada experimentalmente neste trabalho através da análise da correlação espacial modelada pela função de Bessel de primeira espécie.

Diante disso, este trabalho apresenta a implementação e validação experimental de um sistema completo de \ac{PKG} para redes 5G/IoT. A proposta investiga a viabilidade prática e o desempenho do sistema sob diferentes condições de canal e parâmetros de comunicação, incluindo esquemas de modulação digital (\ac{BPSK} e \ac{QPSK}), protocolo de reconciliação \textit{code-offset} com códigos corretores de erro \ac{BCH}(127,64,10) e amplificação de privacidade via SHA-256. Sete experimentos sistemáticos foram conduzidos para avaliar: (i)~o impacto da relação sinal-ruído (\ac{SNR}) na taxa de discordância de chaves (\ac{KDR}), identificando SNR mínimo operacional de 13--15~dB; (ii)~a comparação entre modulações \ac{BPSK} e \ac{QPSK}; (iii)~a influência de diferentes códigos \ac{BCH} na capacidade de reconciliação; (iv)~a complexidade computacional do sistema, demonstrando latência de 0.489~ms para codificação e decodificação BCH, viabilizando aplicação em dispositivos de baixo custo; (v)~a aplicabilidade em cinco perfis IoT distintos---sensores estáticos, \textit{wearables}, veículos urbanos (60~km/h), drones (40~km/h) e NB-IoT; (vi)~a segurança contra espionagem passiva, validando descorrelação espacial a partir de 20~cm de separação; (vii)~o impacto de limiarização adaptativa (\textit{guard-band}) na eficiência e segurança. Os resultados demonstram \ac{KDR} inferior a 0.03\% entre usuários legítimos, taxa de erro de bits (\ac{BER}) de aproximadamente 50\% para o atacante (equivalente a chute aleatório), e viabilidade operacional em cenários de alta mobilidade com correlação temporal de apenas 0.16, indicando que erro de estimação controlado é fator mais crítico que correlação temporal para o sucesso do sistema.

O restante deste artigo está organizado da seguinte forma: a Seção~II apresenta a fundamentação teórica sobre \ac{PKG} e suas etapas principais; a Seção~III descreve o modelo de sistema baseado em correlação espacial de canais; a Seção~IV detalha a metodologia experimental, incluindo os sete experimentos sistemáticos conduzidos; a Seção~V apresenta e discute os resultados numéricos obtidos, incluindo comparação com o estado da arte; e a Seção~VI apresenta as conclusões e perspectivas de trabalhos futuros.