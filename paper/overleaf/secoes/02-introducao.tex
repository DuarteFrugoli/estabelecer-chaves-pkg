\section{Introduction}

A evolução das comunicações móveis, impulsionada pelo desenvolvimento das redes \ac{5G} e pelas pesquisas em \ac{6G}, tem intensificado a demanda por mecanismos de segurança mais eficientes e adaptados ao ambiente sem fio. Com o crescimento do número de dispositivos conectados e da troca de informações sensíveis, garantir a segurança dos dados transmitidos tornou-se um desafio central para pesquisadores e engenheiros de telecomunicações.

Uma comunicação segura deve atender a cinco pilares fundamentais da Segurança da Informação:

\begin{itemize}
    \item \textbf{Confidencialidade:} garante que apenas as partes autorizadas possam ter acesso ao conteúdo transmitido, impedindo a interceptação e o vazamento de dados;
    \item \textbf{Integridade:} assegura que a informação não seja alterada de forma indevida durante o armazenamento ou transmissão, preservando seu conteúdo original;
    \item \textbf{Autenticidade:} assegura que as entidades envolvidas na comunicação são genuínas, evitando ataques de falsificação e impersonificação;
    \item \textbf{Irretratabilidade (ou não-repúdio):} impede que uma das partes envolvidas negue posteriormente ter participado de uma comunicação ou transação;
    \item \textbf{Disponibilidade:} garante que os dados e serviços estejam acessíveis sempre que necessário, evitando interrupções causadas por falhas ou ataques.
\end{itemize}

Dentre esses pilares, a \textbf{confidencialidade} desempenha um papel central na proteção das informações trafegadas no meio sem fio. Sua implementação tradicional é realizada por algoritmos criptográficos, como \ac{AES} e \ac{RSA}, que dependem diretamente do estabelecimento seguro de chaves simétricas ou assimétricas. No entanto, os métodos convencionais de troca de chaves — como Diffie–Hellman e protocolos baseados em certificação — envolvem elevado custo computacional e apresentam vulnerabilidades crescentes diante do avanço da computação quântica.

Esse cenário reforça a necessidade de mecanismos complementares para o estabelecimento de segredos, especialmente em dispositivos móveis e ambientes com recursos restritos. Entre as alternativas emergentes, destaca-se a \textbf{Segurança em Camada Física} (\ac{PLS}), abordagem que explora propriedades físicas intrínsecas do canal sem fio — como desvanecimento, ruído e reciprocidade — para gerar chaves simétricas sem depender de transmissão explícita de chaves, reduzindo o custo computacional e aumentando a resiliência contra ataques.

O presente trabalho tem como objetivo simular, em ambiente Python, um sistema de \textit{Physical Key Generation} (\ac{PKG}) fundamentado nos princípios apresentados no artigo \textit{``Wireless Channel Key Generation for Multi-User Access Scenarios''}, o qual emprega técnicas de reconciliação baseadas em códigos \ac{BCH}. O estudo investiga a influência da distância física entre os dispositivos na taxa de discordância de chaves (\ac{KDR}), avaliando o uso do ganho do canal como parâmetro para geração de segredos e caracterizando os limites físicos que viabilizam a criação de chaves seguras em sistemas móveis de próxima geração. Embora o foco principal esteja nas comunicações móveis, a metodologia proposta também apresenta potencial aplicação em dispositivos embarcados e sistemas \ac{IoT}, nos quais soluções de segurança de baixo custo computacional são particularmente relevantes.

Nas seções seguintes, apresenta-se a fundamentação teórica utilizada, o modelo de simulação adotado e os resultados obtidos.
