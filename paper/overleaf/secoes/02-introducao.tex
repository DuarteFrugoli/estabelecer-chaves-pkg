\section{Introdução}

A evolução das comunicações móveis, impulsionada pelo desenvolvimento das redes \ac{5G} e pelas pesquisas em \ac{6G}, tem intensificado a demanda por mecanismos de segurança mais eficientes e adaptados ao ambiente sem fio. Com o crescimento do número de dispositivos conectados e da troca de informações sensíveis, garantir a segurança dos dados transmitidos tornou-se um desafio central para pesquisadores e engenheiros de telecomunicações \cite{cirani_iot}.

Uma comunicação segura deve atender a cinco pilares fundamentais para prover a segurança da informação \cite{stallings_crypto}. A confidencialidade é o pilar que busca garantir que apenas as partes autorizadas possam ter acesso ao conteúdo transmitido, impedindo a interceptação e o vazamento de dados. A integridade visa assegurar que a informação não seja alterada de forma indevida durante o armazenamento ou transmissão, preservando seu conteúdo original. A autenticidade visa assegurar que as entidades envolvidas na comunicação sejam genuínas, evitando ataques de falsificação e impersonificação. A irretratabilidade (ou não-repúdio) busca impedir que uma das partes envolvidas negue posteriormente ter participado de uma comunicação ou transação. Por fim, a disponibilidade visa garantir que os dados e serviços estejam acessíveis sempre que necessário, evitando interrupções causadas por falhas ou ataques.

Dentre esses pilares, a confidencialidade desempenha um papel central na proteção das informações trafegadas. Os algoritmos criptográficos têm como objetivo transformar a informação original (texto claro) em uma versão cifrada (texto cifrado), de modo que apenas entidades autorizadas possam recuperar o conteúdo original \cite{stallings_crypto}. Para realizar essa transformação, são necessárias chaves criptográficas, que funcionam como parâmetros secretos utilizados tanto no processo de cifragem quanto no de decifragem. No caso da criptografia simétrica, como o \ac{AES}, ambos os usuários da comunicação devem ter conhecimento da mesma chave secreta, sendo fundamental que essa chave não trafegue pelo canal de forma clara, pois sua interceptação comprometeria toda a segurança do sistema. Surge, então, o problema do estabelecimento de um canal seguro para a troca inicial de chaves. Os métodos convencionais para solucionar esse problema incluem o protocolo Diffie–Hellman \cite{diffie_hellman} e infraestruturas de chave pública baseadas em algoritmos assimétricos, como o \ac{RSA}. No entanto, esses métodos apresentam desvantagens relevantes para cenários de redes móveis e \ac{IoT}: envolvem elevado custo computacional, o que limita sua aplicabilidade em dispositivos com recursos restritos, e apresentam vulnerabilidades crescentes diante do avanço da computação quântica \cite{mosca_quantum}.

Esse cenário reforça a necessidade de mecanismos complementares para o estabelecimento de segredos, especialmente em dispositivos móveis e ambientes com capacidade computacional e energética limitadas. Entre as alternativas emergentes, destaca-se a \ac{PLS}, abordagem que explora propriedades físicas intrínsecas do canal de comunicação sem fio — como reciprocidade do canal, variações temporais e ruído térmico — para estabelecer chaves simétricas em ambos os dispositivos que desejam se comunicar de forma confidencial sem depender de transmissão explícita de chaves, reduzindo o custo computacional e aumentando a resiliência contra ataques \cite{bloch_wireless_security, zhou_pls_survey}.

Dentro desse contexto, a \ac{PKG} emerge como uma técnica promissora que permite a dois dispositivos legítimos estabelecerem uma chave criptográfica comum explorando as características aleatórias e recíprocas do canal de comunicação \cite{mathur_pks, zhang_pks_survey}. Diferentemente dos métodos tradicionais que exigem infraestrutura de chave pública ou pré-distribuição de segredos, a \ac{PKG} utiliza as flutuações naturais do canal sem fio para gerar bits de entropia compartilhados entre transmissor e receptor, desde que ambos realizem medições do canal em um curto intervalo de tempo, respeitando o tempo de coerência do canal, e que o ambiente apresente variações suficientes para gerar entropia.

Esta abordagem é particularmente relevante para redes \ac{5G}/\ac{6G} e sistemas \ac{IoT}, onde a escalabilidade, o baixo consumo energético e a maior resiliência contra ameaças quânticas são requisitos críticos \cite{zeng_pkg_challenges}. Em redes densas com milhares de dispositivos conectados simultaneamente, mecanismos que não dependem de infraestrutura centralizada de gerenciamento de chaves tornam-se essenciais para garantir a viabilidade operacional. Adicionalmente, dispositivos \ac{IoT} frequentemente operam com baterias limitadas, tornando inviável o uso intensivo de algoritmos criptográficos pesados. A \ac{PKG}, ao explorar o canal físico já utilizado para comunicação, oferece uma solução energeticamente eficiente, pois as estimações do canal são realizadas durante o próprio processo de comunicação, sem necessidade de transmissões adicionais ou processamento criptográfico complexo, reduzindo significativamente o consumo de energia em comparação com protocolos tradicionais de estabelecimento de chaves \cite{zeng_pkg_challenges}. Por fim, com o avanço dos computadores quânticos, algoritmos tradicionais como \ac{RSA} e Diffie–Hellman tornam-se vulneráveis, enquanto a \ac{PKG}, ao não depender de problemas matemáticos computacionalmente complexos, apresenta maior resiliência a esses ataques \cite{mosca_quantum}.

O presente trabalho apresenta a implementação e análise de um sistema completo de comunicação móvel entre dois dispositivos \ac{IoT} que objetiva \ac{PKG}, investigando sua viabilidade prática e desempenho sob diferentes condições de canal e parâmetros de comunicação, tais como \ac{SNR}, parâmetros de desvanecimento (\(\sigma\)), esquemas de modulação digital (\ac{BPSK} e \ac{QPSK}), e protocolo de reconciliação \textit{code-offset} com suporte de códigos corretores de erro \ac{BCH} \cite{proakis_digital}. Os resultados experimentais demonstram, de forma geral, um bom estabelecimento de chaves criptográficas sob diversas condições de operação.

O restante deste artigo está organizado da seguinte forma: a Seção II apresenta a fundamentação teórica sobre \ac{PKG}, seus componentes principais e o modelo de sistema; a Seção III se refere aos detalhes de implementação e simulação; a Seção IV apresenta os resultados numéricos; e a Seção V apresenta as conclusões.
