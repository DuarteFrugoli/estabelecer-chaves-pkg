\section{Introduction}

A evolução das comunicações móveis, impulsionada pelo desenvolvimento das redes \ac{5G} e pelas pesquisas em \ac{6G}, tem intensificado a demanda por mecanismos de segurança mais eficientes e adaptados ao ambiente sem fio. Com o crescimento do número de dispositivos conectados e da troca de informações sensíveis, garantir a segurança dos dados transmitidos tornou-se um desafio central para pesquisadores e engenheiros de telecomunicações.

Uma comunicação segura deve atender a cinco pilares fundamentais da Segurança da Informação. A \textbf{confidencialidade} garante que apenas as partes autorizadas possam ter acesso ao conteúdo transmitido, impedindo a interceptação e o vazamento de dados. A \textbf{integridade} assegura que a informação não seja alterada de forma indevida durante o armazenamento ou transmissão, preservando seu conteúdo original. A \textbf{autenticidade} assegura que as entidades envolvidas na comunicação são genuínas, evitando ataques de falsificação e impersonificação. A \textbf{irretratabilidade} (ou não-repúdio) impede que uma das partes envolvidas negue posteriormente ter participado de uma comunicação ou transação. Por fim, a \textbf{disponibilidade} garante que os dados e serviços estejam acessíveis sempre que necessário, evitando interrupções causadas por falhas ou ataques.

Dentre esses pilares, a \textbf{confidencialidade} desempenha um papel central na proteção das informações trafegadas no meio sem fio. Sua implementação tradicional é realizada por algoritmos criptográficos, como \ac{AES} e \ac{RSA}, que dependem diretamente do estabelecimento seguro de chaves simétricas ou assimétricas. No entanto, os métodos convencionais de troca de chaves — como Diffie–Hellman e protocolos baseados em certificação — envolvem elevado custo computacional e apresentam vulnerabilidades crescentes diante do avanço da computação quântica.

Esse cenário reforça a necessidade de mecanismos complementares para o estabelecimento de segredos, especialmente em dispositivos móveis e ambientes com recursos restritos. Entre as alternativas emergentes, destaca-se a \textbf{Segurança em Camada Física} (\ac{PLS}), abordagem que explora propriedades físicas intrínsecas do canal sem fio — como desvanecimento, ruído e reciprocidade — para gerar chaves simétricas sem depender de transmissão explícita de chaves, reduzindo o custo computacional e aumentando a resiliência contra ataques.

Dentro desse contexto, o \textit{Physical Key Generation} (\ac{PKG}) emerge como uma técnica promissora que permite a dois dispositivos legítimos estabelecerem uma chave criptográfica comum explorando as características aleatórias e recíprocas do canal de comunicação. Diferentemente dos métodos tradicionais que exigem infraestrutura de chave pública ou pré-distribuição de segredos, o \ac{PKG} utiliza as flutuações naturais do canal sem fio para gerar bits de entropia compartilhados entre transmissor e receptor. Esta abordagem é particularmente relevante para redes \ac{5G}/\ac{6G} e sistemas \ac{IoT}, onde a escalabilidade, o baixo consumo energético e a resiliência contra ameaças quânticas são requisitos críticos.

O presente trabalho apresenta a implementação e análise de um sistema completo de \ac{PKG}, investigando sua viabilidade prática e desempenho sob diferentes condições de canal. Os resultados experimentais demonstram redução média de 22 pontos percentuais na \ac{KDR} após reconciliação, com convergência para 0\% em \ac{SNR} $\geq$ 4dB, validando a eficácia do método proposto.

O restante deste artigo está organizado da seguinte forma: a Seção II apresenta a fundamentação teórica sobre \ac{PKG} e seus componentes principais; a Seção III descreve o modelo de sistema proposto; a Seção IV detalha a implementação e os parâmetros de simulação; e a Seção V apresenta os resultados obtidos e conclusões.
