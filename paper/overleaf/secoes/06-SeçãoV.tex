\section{Resultados e Discussão}
\label{sec:results}

Esta seção apresenta os resultados dos sete experimentos sistemáticos conduzidos para validação do sistema de geração de chaves em camada física. Os experimentos cobrem aspectos de desempenho, segurança, complexidade computacional e aplicabilidade prática em diferentes cenários \ac{IoT}. Todas as simulações empregaram análise de Monte Carlo com 1000 realizações independentes, garantindo significância estatística dos resultados.

\subsection{Experimento 1: Impacto da Relação Sinal-Ruído}

O primeiro experimento investiga o impacto da \ac{SNR} sobre a \ac{BMR} inicial e o \ac{KDR} após reconciliação. A \ac{SNR} foi variada logaritmicamente de $-10$~dB a $30$~dB em 18 pontos, mantendo fixos a correlação espacial $\rho = 0.9$, a modulação \ac{BPSK}, e o código \ac{BCH}(127,64,10).

A Figura~\ref{fig:exp01_snr} apresenta as curvas de \ac{BMR} e \ac{KDR} em função da \ac{SNR}, demonstrando o impacto da reconciliação \ac{BCH}. A Figura~\ref{fig:exp01_bmr} mostra que o \ac{BMR} decresce continuamente com a \ac{SNR}, comportamento esperado uma vez que o aumento da relação sinal-ruído reduz diretamente a probabilidade de erro na detecção dos símbolos transmitidos. Por sua vez, a Figura~\ref{fig:exp01_kdr} evidencia que o \ac{KDR} apresenta decaimento exponencial mais acentuado, atingindo valores nulos a partir de $13$--$15$~dB. Esse comportamento reflete a eficácia do código \ac{BCH}(127,64,10), cuja capacidade de correção de até $t=10$ erros por bloco permite reconciliar completamente as sequências quando o \ac{BMR} inicial é suficientemente baixo. 

Observa-se um fenômeno importante: em regiões de \ac{SNR} muito baixa (tipicamente abaixo de $0$~dB), podem ocorrer casos em que o \ac{KDR} após reconciliação supera ligeiramente o \ac{BMR} inicial. Esse comportamento aparentemente contraditório decorre de falhas de decodificação \ac{BCH}: quando o número de erros por bloco de 127 bits excede a capacidade de correção do código ($>10$ erros), o decodificador pode produzir uma palavra-código incorreta que diverge ainda mais da sequência original, aumentando temporariamente a discrepância. Esse efeito, documentado em trabalhos anteriores sobre códigos lineares~\cite{proakis_digital}, evidencia que o sistema opera adequadamente apenas acima de um \ac{SNR} mínimo onde a taxa de erro inicial é compatível com a capacidade corretiva do código empregado. Para o código \ac{BCH}(127,64,10), esse limiar situa-se em aproximadamente $8$--$9$~dB, como confirmado pela Tabela~\ref{tab:exp01_snr}. A Tabela~\ref{tab:exp01_snr} apresenta os valores numéricos obtidos para pontos selecionados.

\begin{figure*}[t]
\centering
\subfigure[BMR antes da reconciliação]{\includegraphics[width=0.48\textwidth]{figuras/exp01_variacao_snr_01_20260212_021057.png}\label{fig:exp01_bmr}}
\hfill
\subfigure[KDR após reconciliação BCH(127,64,10)]{\includegraphics[width=0.48\textwidth]{figuras/exp01_variacao_snr_02_20260212_021057.png}\label{fig:exp01_kdr}}
\caption{Impacto da SNR no desempenho do sistema de geração de chaves em camada física.}
\label{fig:exp01_snr}
\end{figure*}

\begin{table}[htbp]
\centering
\caption{Desempenho do sistema para valores selecionados de SNR.}
\label{tab:exp01_snr}
\begin{tabular}{cccc}
\hline
\textbf{SNR (dB)} & \textbf{BMR (\%)} & \textbf{KDR (\%)} & \textbf{Status} \\
\hline
$-10.00$ & $44.45$ & $47.49$ & Canal ruidoso \\
$0.00$ & $24.56$ & $37.12$ & Insuficiente \\
$8.82$ & $5.40$ & $2.79$ & Limiar de operação \\
$11.18$ & $3.38$ & $0.03$ & Operacional \\
$13.53$ & $2.06$ & $0.00$ & Ideal \\
$\geq 15.88$ & $\leq 1.27$ & $0.00$ & Ideal \\
\hline
\end{tabular}
\end{table}

Os resultados demonstram que um \ac{SNR} mínimo de $13$--$15$~dB é necessário para operação confiável do sistema, caracterizada por \ac{KDR} nula entre Alice e Bob. Esse requisito é plenamente alcançável em aplicações práticas de \ac{IoT}, sendo compatível com enlaces de comunicação típicos em redes celulares \ac{5G} e \ac{NB-IoT} em ambientes internos e urbanos~\cite{3gpp_nr_coverage, nb_iot_link_budget}, nos quais valores de \ac{SNR} superiores a 15~dB são rotineiramente observados para dispositivos em condições normais de operação.

\subsection{Experimento 2: Comparação entre BPSK e QPSK}

O segundo experimento compara o desempenho dos esquemas de modulação \ac{BPSK} (1~bit/símbolo) e \ac{QPSK} (2~bits/símbolo). Manteve-se correlação espacial $\rho = 0.9$ e variou-se a \ac{SNR} conforme o Experimento~1.

A Tabela~\ref{tab:exp02_modulacao} apresenta a comparação entre as duas modulações para valores selecionados de \ac{SNR}. Os resultados indicam desempenho muito similar entre \ac{BPSK} e \ac{QPSK}, com diferenças marginais de \ac{BMR} e \ac{KDR} que se tornam desprezíveis em regime de alta \ac{SNR}. A equivalência de desempenho é explicada pelo fato de que, embora \ac{QPSK} possua quatro símbolos e portanto maior probabilidade de erro de símbolo, a quantização binária aplicada (mapeamento para $\pm 1$ nas componentes I e Q) resulta em taxas de erro de bit similares para as duas modulações quando operando na mesma \ac{SNR}. Isso demonstra que a segurança e eficiência do sistema de geração de chaves não é significativamente afetada pela escolha entre \ac{BPSK} e \ac{QPSK}, permitindo que outros critérios (eficiência espectral, simplicidade de implementação) guiem essa decisão em aplicações práticas.

\begin{table}[htbp]
\centering
\small
\caption{Comparação de desempenho entre modulação BPSK e QPSK.}
\label{tab:exp02_modulacao}
\begin{tabular}{c|cc|cc}
\hline
\multirow{2}{*}{\textbf{SNR (dB)}} & \multicolumn{2}{c|}{\textbf{BMR (\%)}} & \multicolumn{2}{c}{\textbf{KDR (\%)}} \\
\cline{2-5}
 & \textbf{BPSK} & \textbf{QPSK} & \textbf{BPSK} & \textbf{QPSK} \\
\hline
$8.82$ & $5.55$ & $5.47$ & $2.93$ & $3.37$ \\
$11.18$ & $3.40$ & $3.33$ & $0.15$ & $0.20$ \\
$13.53$ & $2.00$ & $2.03$ & $0.00$ & $0.03$ \\
$30.00$ & $0.06$ & $0.05$ & $0.00$ & $0.00$ \\
\hline
\end{tabular}
\end{table}

A equivalência de desempenho entre \ac{BPSK} e \ac{QPSK} para geração de chaves em camada física é um resultado relevante, indicando que a escolha da modulação pode ser guiada por outros requisitos do sistema (eficiência espectral, simplicidade de implementação) sem impacto significativo na qualidade das chaves geradas. Do ponto de vista de segurança, ambas as modulações fornecem nível equivalente de proteção, uma vez que a segurança do sistema deriva fundamentalmente da descorrelação espacial do canal e não da modulação empregada.

\subsection{Experimento 3: Variação do Código BCH}

O terceiro experimento avalia o desempenho de diferentes códigos \ac{BCH} na etapa de reconciliação: \ac{BCH}(7,4,1), \ac{BCH}(15,7,2) e \ac{BCH}(127,64,10). A Tabela~\ref{tab:exp03_bch} apresenta os resultados obtidos para \ac{SNR} de $11.18$~dB.

\begin{table}[htbp]
\centering
\caption{Comparação de desempenho entre diferentes códigos BCH (SNR = 11.18~dB).}
\label{tab:exp03_bch}
\begin{tabular}{cccccc}
\hline
\textbf{Código} & \textbf{n} & \textbf{k} & \textbf{t} & \textbf{Taxa} & \textbf{KDR (\%)} \\
\hline
BCH(7,4) & 7 & 4 & 1 & 0.57 & 0.77 \\
BCH(15,7) & 15 & 7 & 2 & 0.47 & 0.21 \\
BCH(127,64) & 127 & 64 & 10 & 0.50 & \textbf{0.09} \\
\hline
\end{tabular}
\end{table}

Os resultados confirmam que o código \ac{BCH}(127,64,10), com capacidade de correção de até $t=10$ erros, oferece o melhor compromisso entre robustez e eficiência (taxa de código $R = 0.50$), justificando sua escolha para o sistema proposto. A maior capacidade de correção de erros do código \ac{BCH}(127,64,10) traduz-se diretamente em maior segurança do sistema, pois permite operação confiável (\ac{KDR} $\approx 0\%$) em condições de canal mais adversas, reduzindo a necessidade de retransmissões que poderiam expor informação adicional ao atacante

\subsection{Experimento 4: Análise de Complexidade Computacional}

O quarto experimento mede o tempo de execução das principais operações do sistema para diferentes códigos \ac{BCH}. A Figura~\ref{fig:exp04_complexidade} apresenta graficamente os tempos de codificação e decodificação. Observa-se que o código \ac{BCH}(127,64,10) apresenta tempo total de processamento de 0.489~ms (0.061~ms para codificação e 0.428~ms para decodificação), demonstrando viabilidade para aplicações \ac{IoT} em tempo real que tipicamente exigem latências inferiores a 10~ms. A Tabela~\ref{tab:exp04_complexidade} apresenta os valores numéricos completos.

\begin{figure}[htbp]
\centering
\includegraphics[width=0.85\columnwidth]{figuras/exp04_analise_complexidade_01_20260212_021634.png}
\caption{Complexidade computacional: tempo de codificação versus decodificação para diferentes códigos BCH.}
\label{fig:exp04_complexidade}
\end{figure}

\begin{table}[htbp]
\centering
\caption{Análise de complexidade computacional para diferentes códigos BCH.}
\label{tab:exp04_complexidade}
\begin{tabular}{cccc}
\hline
\textbf{Código} & \textbf{Codificação (ms)} & \textbf{Decodificação (ms)} & \textbf{Total (ms)} \\
\hline
BCH(7,4) & 0.042 & 0.126 & 0.168 \\
BCH(15,7) & 0.061 & 0.163 & 0.224 \\
BCH(127,64) & 0.061 & 0.428 & \textbf{0.489} \\
BCH(255,139) & 0.052 & 0.447 & 0.499 \\
\hline
\end{tabular}
\end{table}

O tempo de processamento de $0.489$~ms para o código \ac{BCH}(127,64) corresponde a uma capacidade teórica de aproximadamente $2044$ operações por segundo, demonstrando a viabilidade do sistema para aplicações de \ac{IoT} em tempo real, mesmo em dispositivos de baixo custo operando em software.

\subsection{Experimento 5: Perfis de Dispositivos IoT}

O quinto experimento valida a aplicabilidade do sistema em cinco perfis representativos de dispositivos \ac{IoT}, cada um caracterizado por parâmetros físicos realistas de mobilidade, frequência de operação e condições de canal. A Figura~\ref{fig:exp05_perfis} apresenta graficamente o comportamento de \ac{BMR} e \ac{KDR} para todos os perfis. A Figura~\ref{fig:exp05_bmr} mostra que o \ac{BMR} inicial varia entre os perfis de acordo com o erro de estimação de canal adotado (8\% para sensor estático, 30\% para veículo urbano). Já a Figura~\ref{fig:exp05_kdr} demonstra que todos os perfis atingem \ac{KDR} inferior a 1\% em 9--11~dB e \ac{KDR} nula em 13~dB, evidenciando robustez do sistema desde sensores estáticos até veículos urbanos a 60~km/h. A Tabela~\ref{tab:exp05_perfis} sumariza os parâmetros e \ac{SNR} mínimo operacional.

\begin{table}[htbp]
\centering
\caption{Perfis de dispositivos IoT e SNR mínimo operacional.}
\label{tab:exp05_perfis}
\begin{tabular}{lcccc}
\hline
\textbf{Perfil} & \textbf{Velocidade} & \textbf{Frequência} & \textbf{$\rho_{\text{temporal}}$} & \textbf{SNR$_{\text{min}}$ (dB)} \\
\hline
Sensor estático & 0~km/h & 868~MHz & 1.000 & 9 \\
Pessoa andando & 5~km/h & 2.4~GHz & 0.940 & 11 \\
Veículo urbano & 60~km/h & 5.9~GHz & 0.160 & 11 \\
Drone & 40~km/h & 2.4~GHz & 0.609 & 11 \\
NB-IoT & 10~km/h & 900~MHz & 0.955 & 11 \\
\hline
\end{tabular}
\end{table}

Observa-se que o perfil de sensor estático apresenta o melhor desempenho, atingindo \ac{KDR}~$<1\%$ em apenas $9$~dB devido à correlação temporal perfeita ($\rho=1.0$) e erro de estimação baixo ($8\%$). Os demais perfis convergem para \ac{KDR}~$<1\%$ em $11$~dB, e todos alcançam \ac{KDR} nula em $13$~dB.

Um resultado notável é a operação bem-sucedida do sistema no cenário de veículo urbano ($60$~km/h, $\rho_{\text{temporal}} = 0.16$), demonstrando que o erro de estimação de canal controlado ($\leq 30\%$) é mais crítico para a viabilidade do sistema do que a correlação temporal propriamente dita. Esse insight sugere que estratégias de estimação robusta de canal são fundamentais para extensão do sistema a cenários de alta mobilidade. Do ponto de vista de segurança, a validação em múltiplos perfis com \ac{SNR} mínimo entre 9--13~dB garante que o sistema opera de forma confiável em condições práticas diversas, mantendo \ac{KDR} nula e, portanto, impedindo que Eve obtenha qualquer informação útil sobre as chaves estabelecidas entre Alice e Bob.

\begin{figure*}[t]
\centering
\subfigure[BMR antes da reconciliação]{\includegraphics[width=0.48\textwidth]{figuras/exp05_perfis_dispositivos_20260212_021230_01.png}\label{fig:exp05_bmr}}
\hfill
\subfigure[KDR após reconciliação BCH(127,64,10)]{\includegraphics[width=0.48\textwidth]{figuras/exp05_perfis_dispositivos_20260212_021230_02.png}\label{fig:exp05_kdr}}
\caption{Desempenho do sistema em cinco perfis representativos de dispositivos IoT.}
\label{fig:exp05_perfis}
\end{figure*}

\subsection{Experimento 6: Análise de Segurança contra Espionagem Passiva}

O sexto experimento investiga a segurança do sistema contra um atacante passivo (Eve) que observa o canal a partir de posições espaciais e temporais distintas. Foram avaliadas duas configurações: descorrelação espacial e descorrelação temporal.

\subsubsection{Descorrelação Espacial}

A Tabela~\ref{tab:exp06_espacial} apresenta a correlação espacial de Eve em relação a Alice e a correspondente \ac{BER} observada por Eve para diferentes distâncias laterais. A frequência de operação foi fixada em $2.4$~GHz ($\lambda = 12.5$~cm) e a \ac{SNR} em $9$~dB (cenário de teste de estresse em condições moderadamente ruidosas).

\begin{table}[htbp]
\centering
\caption{Segurança contra espionagem em função da descorrelação espacial.}
\label{tab:exp06_espacial}
\begin{tabular}{cccc}
\hline
\textbf{Distância Eve (m)} & \textbf{$\lambda/2$} & \textbf{$\rho_{\text{espacial}}$} & \textbf{BER Eve (\%)} \\
\hline
0.10 & 1.6 & 0.210 & 48.46 \\
0.20 & 3.2 & 0.020 & 48.46 \\
0.50 & 8.0 & 0.002 & 48.65 \\
$\geq 1.00$ & $\geq 16$ & $\approx 0$ & $\approx 48.5$ \\
\hline
\end{tabular}
\end{table}

Os resultados demonstram que para distâncias superiores a $0.2$~m (aproximadamente $3.2\lambda/2$), a correlação espacial torna-se desprezível ($\rho < 0.02$), garantindo \ac{BER} de Eve próxima a $50\%$, equivalente a uma tentativa de adivinhação aleatória. Esse comportamento valida experimentalmente o modelo de Clarke~\cite{clarke_1968} e confirma a segurança física do sistema contra espionagem passiva. A Figura~\ref{fig:exp06_eve} ilustra graficamente o decaimento da correlação espacial $\rho(h_{\text{Alice}}, h_{\text{Eve}})$ em função da distância lateral, calculada pelo método de Yuan et al.~\cite{yuan_fast_2013}. Observa-se que a partir de 20~cm ($\approx 3\lambda/2$ a 2.4~GHz), a correlação torna-se desprezível ($\rho < 0.02$), garantindo que Eve não consiga extrair informação útil sobre a chave gerada.

\begin{figure}[htbp]
\centering
\includegraphics[width=0.75\columnwidth]{figuras/exp06_analise_eve_20260212_021230_01.png}
\caption{Descorrelação espacial: correlação de canal entre Alice e Eve versus distância lateral.}
\label{fig:exp06_eve}
\end{figure}

\subsubsection{Descorrelação Temporal}

A análise de descorrelação temporal, realizada mantendo Eve a uma distância fixa de $0.5$~m (já descorrelacionada espacialmente), demonstrou que a correlação temporal é irrelevante para a segurança quando a descorrelação espacial é suficiente. Mesmo com atrasos temporais variando de $0$ a $10$~ms, a correlação total $\rho_{\text{total}} = \rho_{\text{espacial}} \times \rho_{\text{temporal}}$ permaneceu próxima a zero devido ao termo espacial dominante.

\subsection{Experimento 7: Impacto do Guard-Band}

O sétimo e último experimento avalia o impacto da limiarização adaptativa (\textit{guard-band}) sobre a eficiência e segurança do sistema. Variou-se o parâmetro de \textit{guard-band} de $0.0$ (sem zona morta) a $1.0$ (zona morta ampla, em múltiplos de $\sigma_n$), mantendo \ac{SNR} fixa em $15$~dB.

A Tabela~\ref{tab:exp07_guardband} apresenta os resultados obtidos. Observa-se que o sistema apresenta \ac{BER} de Eve próxima a $50\%$ mesmo sem utilização de \textit{guard-band} (GB $= 0$), indicando que o sistema é naturalmente seguro devido à descorrelação espacial. Surpreendentemente, valores elevados de \textit{guard-band} (GB $\geq 0.5$) não apenas reduzem drasticamente a taxa efetiva de geração de chaves (devido ao descarte de bits próximos ao limiar), mas também podem introduzir viés estatístico que degrada ligeiramente a segurança (BER Eve $= 46.82\%$ para GB $= 1.0$).

\begin{table}[htbp]
\centering
\caption{Impacto do guard-band na eficiência e segurança do sistema.}
\label{tab:exp07_guardband}
\begin{tabular}{ccccc}
\hline
\textbf{GB ($\sigma$)} & \textbf{KDR Bob (\%)} & \textbf{BER Eve (\%)} & \textbf{Taxa (kbps)} & \textbf{Descarte (\%)} \\
\hline
0.0 & 0.03 & 49.67 & 127.0 & 0.0 \\
0.1 & 0.00 & 49.92 & 118.4 & 6.8 \\
0.3 & 0.03 & 50.07 & 102.9 & 18.9 \\
0.5 & 0.02 & 49.91 & 89.5 & 29.5 \\
1.0 & 0.03 & 46.82 & 63.1 & 50.3 \\
\hline
\end{tabular}
\end{table}

\begin{figure}[htbp]
\centering
\includegraphics[width=0.85\columnwidth]{figuras/exp07_impacto_guard_band_20260212_021342_02.png}
\caption{BER de Eve versus guard-band (SNR = 15~dB).}
\label{fig:exp07_guardband}
\end{figure}

Esse resultado é particularmente relevante, pois demonstra que técnicas de limiarização adaptativa complexa, frequentemente propostas na literatura, não são necessárias para garantir a segurança do sistema quando a descorrelação espacial é adequadamente explorada. A Figura~\ref{fig:exp07_guardband} confirma que a \ac{BER} de Eve permanece próxima a $50\%$ (chute aleatório) independentemente do guard-band utilizado, mesmo sem limiarização (GB=0). Isso evidencia que a descorrelação espacial é suficiente para garantir a segurança física do sistema sem necessidade de técnicas adaptativas complexas. Recomenda-se operar o sistema com GB $= 0$ ou GB $\leq 0.1\sigma$ para maximizar a taxa de geração de chaves sem comprometer a segurança.

\subsection{Discussão Geral e Comparação com Estado da Arte}

Os resultados experimentais demonstram a viabilidade técnica do sistema proposto para geração de chaves criptográficas em camada física em cenários práticos de \ac{IoT}. O \ac{SNR} mínimo de $13$--$15$~dB identificado no Experimento~1 é plenamente alcançável em aplicações reais, sendo compatível com enlaces de comunicação típicos em redes celulares \ac{5G} e \ac{NB-IoT}~\cite{3gpp_nr_coverage, nb_iot_link_budget}, onde valores superiores a 15~dB são rotineiramente observados em condições normais de operação.

A complexidade computacional de $0.489$~ms (Experimento~4) viabiliza a implementação em dispositivos de baixo custo operando em software, sem necessidade de hardware dedicado (FPGA, USRP), distinguindo este trabalho de abordagens anteriores que requerem plataformas especializadas~\cite{yuan_fast_2013}. Essa característica é fundamental para massificação do sistema em redes \ac{IoT}, onde bilhões de dispositivos de baixo custo precisam estabelecer chaves de forma eficiente e segura.

A validação em cinco perfis de dispositivos \ac{IoT} (Experimento~5) demonstra a ampla aplicabilidade do sistema, cobrindo desde sensores estáticos até veículos urbanos a $60$~km/h. O funcionamento em cenários de alta mobilidade ($\rho_{\text{temporal}} = 0.16$) sugere que o erro de estimação de canal, quando controlado adequadamente, é mais crítico que a correlação temporal, um insight relevante para projeto de sistemas práticos que contradiz pressupostos comuns na literatura que enfatizam excessivamente a necessidade de alta correlação temporal~\cite{mathur_pks}.

A análise de segurança (Experimento~6) confirma a proteção contra espionagem passiva a partir de distâncias de $20$~cm, validando experimentalmente o modelo teórico de Clarke~\cite{clarke_1968}. A \ac{BER} de Eve próxima a 50\% representa segurança informacional perfeita no sentido teórico da informação~\cite{maurer_secret_key}, uma vez que Eve não obtém nenhuma informação útil sobre a chave estabelecida, mesmo com capacidade computacional ilimitada. Esse nível de segurança contrasta favoravelmente com sistemas criptográficos tradicionais, cuja proteção baseia-se em hipóteses de complexidade computacional vulneráveis a ataques quânticos.

Finalmente, a demonstração de que \textit{guard-band} não é necessário (Experimento~7) representa uma contribuição original, simplificando a implementação prática do sistema sem comprometer a segurança. Esse resultado desafia trabalhos anteriores que propõem mecanismos complexos de limiarização adaptativa~\cite{zeng_pkg_challenges}, demonstrando que em sistemas baseados em correlação espacial (ao invés de reciprocidade temporal), a descorrelação natural do canal é suficiente para garantir segurança física robusta.