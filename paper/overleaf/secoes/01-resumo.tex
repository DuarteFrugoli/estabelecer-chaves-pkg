%
\begin{resumo}
Este trabalho apresenta a implementação e a análise de um sistema completo de comunicação móvel voltado à geração de chaves criptográficas na camada física para comunicações móveis de próxima geração. A abordagem proposta explora propriedades físicas intrínsecas do canal sem fio, tais como reciprocidade, desvanecimento e ruído, para permitir a geração de chaves simétricas idênticas entre dispositivos legítimos, eliminando a necessidade de troca prévia de segredos e reduzindo a dependência de infraestruturas de chave pública. O sistema integra modulação digital, um protocolo de reconciliação do tipo \textit{code-offset} baseado em códigos corretores de erro, além de um estágio de amplificação de privacidade. A análise concentra-se na taxa de discordância entre as chaves criptográficas geradas nos dispositivos comunicantes em função de parâmetros fundamentais do sistema, como a relação sinal-ruído, a frequência de operação e as características do desvanecimento, visando identificar os cenários mais favoráveis para a reconciliação eficiente da chave simétrica. Os resultados indicam que o sistema é aplicável tanto a redes 5G/6G quanto a ambientes de Internet das Coisas de baixo consumo, oferecendo segurança na camada física com \textit{overhead} computacional reduzido em comparação com métodos criptográficos convencionais.
\end{resumo}

\begin{palavraschave}
confidencialidade, segurança em camada física, geração de chaves criptográficas, reconciliação da informação, comunicações móveis.
\end{palavraschave}

\begin{abstract}
This paper presents the implementation and analysis of a complete mobile communication system aimed at physical-layer key generation for next-generation wireless communications. The proposed approach exploits intrinsic physical properties of the wireless channel, such as reciprocity, fading, and noise, to enable the generation of identical symmetric keys between legitimate devices, eliminating the need for prior secret exchange and reducing reliance on public key infrastructures. The system integrates digital modulation, a code-offset reconciliation protocol based on error-correcting codes, as well as a privacy amplification stage. The analysis focuses on the key disagreement rate between the cryptographic keys generated at the communicating devices as a function of fundamental system parameters, including signal-to-noise ratio, operating frequency, and fading characteristics, with the objective of identifying the most favorable scenarios for efficient symmetric key reconciliation. The proposed system is applicable to both 5G/6G networks and low-power Internet of Things environments, providing physical-layer security with reduced computational overhead compared to conventional cryptographic methods.
\end{abstract}

\begin{IEEEkeywords}
Confidentiality, physical layer security, cryptographic key generation, information reconciliation, mobile communications.
\end{IEEEkeywords}
