%
\begin{resumo}
Este trabalho apresenta a implementação e validação experimental de um sistema de geração de chaves criptográficas em camada física para redes 5G e Internet das Coisas. O sistema explora a correlação espacial de canais sem fio observados por dispositivos legítimos próximos conectados à mesma estação base, diferenciando-se de abordagens convencionais baseadas em criptografia assimétrica computacionalmente onerosa. A arquitetura proposta integra modulação digital (BPSK/QPSK), protocolo de reconciliação \textit{code-offset} com códigos BCH, e amplificação de privacidade via \ac{SHA}-256, formando um sistema completo de estabelecimento de chaves independente de infraestrutura centralizada. Sete experimentos sistemáticos validam o desempenho, segurança e viabilidade prática em diferentes condições de canal, esquemas de modulação, códigos corretores de erro e perfis de dispositivos \ac{IoT} representativos. Os resultados demonstram SNR mínimo operacional de 13--15~dB, baixa complexidade computacional compatível com dispositivos de baixo custo, segurança física contra espionagem passiva a partir de 20~cm de separação (BER Eve $\approx$ 50\%), e robustez em cenários de alta mobilidade. O sistema constitui alternativa energeticamente eficiente aos métodos criptográficos convencionais para estabelecimento de chaves em redes de próxima geração.
\end{resumo}

\begin{palavraschave}
Geração de chaves em camada física, segurança 5G/IoT, códigos BCH, reconciliação \textit{code-offset}, correlação espacial.
\end{palavraschave}

\begin{abstract}
This work presents the implementation and experimental validation of a physical-layer key generation system for 5G and Internet of Things networks. The system exploits spatial correlation of wireless channels observed by nearby legitimate devices connected to the same base station, differing from conventional approaches based on computationally expensive asymmetric cryptography. The proposed architecture integrates digital modulation (BPSK/QPSK), \textit{code-offset} reconciliation protocol with BCH codes, and privacy amplification via SHA-256, forming a complete key establishment system independent of centralized infrastructure. Seven systematic experiments validate performance, security, and practical viability under different channel conditions, modulation schemes, error-correction codes, and representative IoT device profiles. Results demonstrate minimum operational SNR of 13--15~dB, low computational complexity compatible with low-cost devices, physical security against passive eavesdropping beyond 20~cm separation (Eve's BER $\approx$ 50\%), and robustness in high-mobility scenarios. The system constitutes an energy-efficient alternative to conventional cryptographic methods for key establishment in next-generation networks.
\end{abstract}

\begin{IEEEkeywords}
Physical-layer key generation, 5G/IoT security, BCH codes, code-offset reconciliation, spatial correlation.
\end{IEEEkeywords}
