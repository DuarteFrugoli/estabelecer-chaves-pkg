%
\begin{resumo}
Este trabalho apresenta a implementação e análise de um sistema completo de comunicação móvel que objetiva a geração de chaves criptográficas via camada física (\textit{Physical Layer Key Generation}) em comunicações móveis de próxima geração. A abordagem proposta explora propriedades físicas intrínsecas do canal de comunicação sem fio, como reciprocidade, desvanecimento e ruído, para gerar chaves simétricas idênticas entre dispositivos legítimos, eliminando a necessidade de troca prévia de segredos e reduzindo a dependência de infraestruturas de chave pública. O sistema integra modulação digital, protocolo de reconciliação \textit{code-offset} com códigos corretores de erro Bose-Chaudhuri-Hocquenghem, bem como amplificação de privacidade. O objetivo é analisar a taxa de discordância de chaves criptográficas em função de parâmetros fundamentais, como relação sinal-ruído, frequência de operação e parâmetros de desvanecimento, buscando identificar o melhor cenário no qual dois dispositivos conseguem reconciliar a informação de chave criptográfica simétrica. O sistema apresenta aplicabilidade tanto em redes 5G/6G quanto em ambientes de Internet das Coisas de baixo consumo, oferecendo segurança de camada física com \textit{overhead} computacional reduzido comparado aos métodos criptográficos tradicionais.
\end{resumo}

\begin{palavraschave}
confidencialidade, segurança em camada física, geração de chaves criptográficas, reconciliação da informação, comunicações móveis.
\end{palavraschave}

\begin{abstract}
This work presents the implementation and analysis of a complete mobile communication system that aims at cryptographic key generation via physical layer (Physical Layer Key Generation) in next-generation mobile communications. The proposed approach takes advantage of the intrinsic physical properties of the wireless channel, such as reciprocity, fading, and noise, to generate identical symmetric keys between legitimate devices, eliminating the need for prior secret exchange and reducing reliance on public key infrastructures. The system integrates digital modulation, code-offset reconciliation protocol with Bose-Chaudhuri-Hocquenghem error correction codes, as well as privacy amplification. The objective is to analyze the Key Disagreement Rate as a function of fundamental parameters such as signal-to-noise ratio, operating frequency, and fading parameters, seeking to identify the best scenario in which two devices can reconcile symmetric cryptographic key information. The system shows applicability in both 5G/6G networks and low-power Internet of Things environments, offering physical layer security with reduced computational \textit{overhead} compared to traditional cryptographic methods.
\end{abstract}

\begin{IEEEkeywords}
Confidentiality, physical layer security, cryptographic key generation, information reconciliation, mobile communications.
\end{IEEEkeywords}
