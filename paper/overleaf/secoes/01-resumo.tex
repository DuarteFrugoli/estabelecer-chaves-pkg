%
\begin{resumo}
Este trabalho apresenta a implementação e validação experimental de um sistema de geração de chaves criptográficas em camada física para redes 5G e Internet das Coisas, explorando correlação espacial de canais sem fio observados por dispositivos legítimos conectados à mesma estação base. O sistema integra modulação BPSK/QPSK, protocolo de reconciliação \textit{code-offset} com códigos BCH(127,64,10) e amplificação de privacidade via SHA-256. Sete experimentos sistemáticos demonstram: (i) SNR mínimo de 13--15~dB garante taxa de discordância de chaves (KDR) nula entre usuários legítimos; (ii) complexidade computacional de 0.489~ms para codificação e decodificação BCH viabiliza aplicação em dispositivos IoT de baixo custo; (iii) alta capacidade de correção do código BCH (t=10) permite KDR inferior a 0.03\% sem necessidade de limiarização adaptativa complexa, maximizando a eficiência de geração de chaves; (iv) descorrelação espacial garante segurança contra espionagem passiva a partir de 20~cm de separação (BER Eve $\approx$ 50\%); (v) validação em cinco perfis IoT distintos---sensores estáticos, \textit{wearables}, veículos urbanos (60~km/h), drones (40~km/h) e NB-IoT---revela que o sistema opera mesmo em cenários de alta mobilidade com correlação temporal reduzida ($\rho=0.16$), indicando que erro de estimação controlado é fator mais crítico que correlação temporal para viabilidade prática. Os resultados demonstram alternativa energeticamente eficiente aos métodos criptográficos convencionais para estabelecimento de chaves em redes de próxima geração.
\end{resumo}

\begin{palavraschave}
Geração de chaves em camada física, segurança 5G/IoT, códigos BCH, reconciliação \textit{code-offset}, correlação espacial.
\end{palavraschave}

\begin{abstract}
This work presents the implementation and experimental validation of a physical-layer key generation system for 5G and Internet of Things networks, exploiting spatial correlation of wireless channels observed by legitimate devices connected to the same base station. The system integrates BPSK/QPSK modulation, \textit{code-offset} reconciliation protocol with BCH(127,64,10) codes, and privacy amplification via SHA-256. Seven systematic experiments demonstrate: (i) minimum SNR of 13--15~dB ensures zero key disagreement rate (KDR) between legitimate users; (ii) computational complexity of 0.489~ms for BCH encoding and decoding enables deployment on low-cost IoT devices; (iii) high error-correction capability of BCH code (t=10) achieves KDR below 0.03\% without requiring complex adaptive thresholding, maximizing key generation efficiency; (iv) spatial decorrelation ensures security against passive eavesdropping beyond 20~cm separation (Eve's BER $\approx$ 50\%); (v) validation across five distinct IoT profiles---static sensors, wearables, urban vehicles (60~km/h), drones (40~km/h), and NB-IoT---reveals that the system operates even in high-mobility scenarios with reduced temporal correlation ($\rho=0.16$), indicating that controlled estimation error is more critical than temporal correlation for practical viability. Results demonstrate an energy-efficient alternative to conventional cryptographic methods for key establishment in next-generation networks.
\end{abstract}

\begin{IEEEkeywords}
Physical-layer key generation, 5G/IoT security, BCH codes, code-offset reconciliation, spatial correlation.
\end{IEEEkeywords}
