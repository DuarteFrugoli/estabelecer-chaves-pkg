%
\begin{resumo}
Este trabalho apresenta a implementação e análise de um sistema completo de comunicação móvel que objetiva PKG para o estabelecimento de chaves criptográficas em comunicações móveis de próxima geração. A abordagem proposta explora propriedades físicas intrínsecas do canal de comunicação sem fio, como reciprocidade, desvanecimento e ruído, para gerar chaves simétricas idênticas entre dispositivos legítimos, eliminando a necessidade de troca prévia de segredos e reduzindo a dependência de infraestruturas de chave pública. O sistema integra modulação digital, códigos corretores de erro para reconciliação da informação, e amplificação de privacidade. O objetivo é analisar o percentual de estabelecimento de chave criptográfica em camada física, em função de parâmetros fundamentais, como SNR, frequência de operação e parâmetros de desvanecimento, buscando intuir sobre o melhor cenário ao qual dois dispositivos conseguem reconciliar a informação. O sistema apresenta aplicabilidade tanto em redes 5G/6G quanto em ambientes IoT de baixo consumo, oferecendo segurança de camada física com \textit{overhead} computacional reduzido comparado aos métodos criptográficos tradicionais.
\end{resumo}

\begin{palavraschave}
confidencialidade, segurança em camada física, geração de chaves criptográficas, reconciliação, comunicações móveis.
\end{palavraschave}

\begin{abstract}
This work presents the implementation and analysis of a complete mobile communication system that aims PKG for establishing cryptographic keys in next-generation mobile communications. The proposed approach exploits intrinsic physical properties of the wireless channel, such as reciprocity, fading, and noise, to generate identical symmetric keys between legitimate devices, eliminating the need for prior secret exchange and reducing reliance on public key infrastructures. The system integrates digital modulation, error correction codes for information reconciliation, and privacy amplification. The objective is to analyze the percentage of cryptographic key establishment at the physical layer, as a function of fundamental parameters such as SNR, operating frequency, and fading parameters, seeking to infer the best scenario in which two devices can reconcile information. The system shows applicability in both 5G/6G networks and low-power IoT environments, offering physical layer security with reduced computational \textit{overhead} compared to traditional cryptographic methods.
\end{abstract}

\begin{IEEEkeywords}
Confidentiality, physical layer security, cryptographic key generation, reconciliation, mobile communications.
\end{IEEEkeywords}
