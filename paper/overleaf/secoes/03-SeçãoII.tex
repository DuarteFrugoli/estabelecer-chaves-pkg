%
%
\section{Geração de Chaves Criptográficas em Camada Física: Fundamentos Teóricos}

A segurança da informação em sistemas de comunicação sem fio tem sido historicamente fundamentada no uso de protocolos de comunicação segura implementados nas camadas superiores da pilha de protocolos. Esses protocolos têm como objetivo garantir, total ou parcialmente, os pilares da comunicação segura, com destaque para a confidencialidade, integridade, autenticidade, irretratabilidade e disponibilidade. Atualmente, tais requisitos são atendidos predominantemente por meio de protocolos amplamente consolidados, como o \ac{TLS} (\textit{Transport Layer Security}), que atua na camada de transporte \cite{rfc_tls}, e o \ac{IPsec} (\textit{Internet Protocol Security}), implementado na camada de rede \cite{rfc_ipsec}. Ambos os protocolos contemplam explicitamente os pilares de confidencialidade, integridade e autenticidade, por meio do uso de algoritmos criptográficos e de mecanismos formais de estabelecimento de chaves, permitindo prover comunicação segura entre entidades legítimas.

Dentre esses pilares, a confidencialidade ocupa um papel central, uma vez que está diretamente relacionada à proteção do conteúdo transmitido contra interceptação e acesso não autorizado. Em protocolos de comunicação segura amplamente adotados, como o \ac{TLS}, a confidencialidade é garantida por meio do uso combinado de algoritmos criptográficos simétricos e assimétricos, cujo funcionamento está intrinsecamente associado a mecanismos de estabelecimento seguro de chaves criptográficas. 

Mecanismos tradicionais de estabelecimento de chaves, como aqueles baseados no protocolo Diffie--Hellman e em infraestruturas de chave pública, embora matematicamente robustos, impõem elevado custo computacional e dependem de entidades de confiança, como autoridades certificadoras, responsáveis pela autenticação das partes envolvidas \cite{diffie_hellman, stallings_crypto}. Essas características tornam-se particularmente críticas em cenários nos quais há grande número de dispositivos, mobilidade elevada e restrições de processamento e energia, motivando a investigação de alternativas ao modelo criptográfico tradicional em ambientes sem fio \cite{zeng_pkg_challenges}.

Nesse contexto, torna-se necessário investigar abordagens complementares para o estabelecimento de chaves criptográficas que reduzam a dependência de camadas superiores e de infraestruturas complexas. A \ac{PLS} surge como uma alternativa promissora ao explorar propriedades intrínsecas do meio físico de propagação, tais como a reciprocidade do canal e o desvanecimento do sinal, para a geração de segredos compartilhados \cite{bloch_wireless_security, zhou_pls_survey}. Trabalhos pioneiros demonstraram que essas características podem ser exploradas para extrair chaves criptográficas diretamente do canal sem fio, reduzindo a dependência exclusiva de pressupostos baseados em complexidade computacional \cite{mathur_pks}. Dessa forma, esta seção apresenta os fundamentos teóricos necessários para compreender o processo de \ac{PKG}, abordando os conceitos de criptografia simétrica e assimétrica, o modelo de sistema adotado, as propriedades do canal sem fio e as etapas envolvidas na geração, reconciliação da informação e amplificação de privacidade das chaves em camada física.

\subsection{Criptografia Simétrica e Assimétrica}

A criptografia é o principal mecanismo empregado para garantir a \textbf{confidencialidade} das informações transmitidas em sistemas de comunicação digitais, sendo também utilizada como base para mecanismos que asseguram integridade e autenticidade. De forma geral, os esquemas criptográficos podem ser classificados em duas grandes classes: criptografia simétrica e criptografia assimétrica. A distinção entre essas classes está diretamente relacionada à forma como as chaves criptográficas são estabelecidas, distribuídas e utilizadas pelas entidades legítimas do sistema.

Na criptografia simétrica, uma única chave secreta é compartilhada entre o transmissor e o receptor, sendo utilizada tanto no processo de cifragem quanto no de decifragem da informação. Algoritmos clássicos dessa categoria incluem o \ac{DES}, seu sucessor \ac{TDES}, e, mais recentemente, o \ac{AES}, que se tornou o padrão amplamente adotado em sistemas modernos devido à sua elevada eficiência computacional e robustez criptográfica \cite{stallings_crypto}. A principal vantagem da criptografia simétrica reside no baixo custo computacional, o que a torna particularmente adequada para aplicações em tempo real e dispositivos com recursos limitados.

Entretanto, a segurança dos esquemas simétricos depende diretamente do compartilhamento prévio e seguro da chave secreta entre as partes envolvidas. Esse requisito caracteriza o chamado problema de distribuição de chaves, no qual é necessário assumir a existência de um mecanismo confiável para o estabelecimento inicial da chave secreta. Em ambientes de comunicação reais, especialmente em redes abertas e sem fio, tal suposição é, em geral, difícil de ser atendida, tornando o estabelecimento seguro de chaves um desafio fundamental.

A criptografia assimétrica foi introduzida com o objetivo de mitigar esse problema. Nesse modelo, cada entidade possui um par de chaves matematicamente relacionadas: uma chave pública, que pode ser amplamente divulgada, e uma chave privada, mantida sob sigilo. Um exemplo amplamente utilizado é o algoritmo RSA, cuja segurança baseia-se na dificuldade computacional do problema da fatoração de grandes números inteiros. Além disso, protocolos de estabelecimento de chaves, como o Diffie--Hellman, permitem que duas entidades concordem sobre um segredo compartilhado sem a necessidade de um canal previamente seguro, explorando a dificuldade do cálculo do logaritmo discreto \cite{diffie_hellman}.

Embora eficientes para o estabelecimento inicial de chaves, mecanismos baseados em criptografia assimétrica apresentam limitações relevantes. O elevado custo computacional inviabiliza sua aplicação direta na proteção contínua de grandes volumes de dados. Ademais, protocolos de estabelecimento de chaves, como o Diffie--Hellman, não fornecem autenticação por si só, o que os torna vulneráveis a ataques do tipo intermediário (\textit{man-in-the-middle}) quando utilizados isoladamente.

Na prática, a autenticação das entidades envolvidas é frequentemente realizada por meio de infraestruturas de chave pública (\textit{Public Key Infrastructure} -- PKI), nas quais autoridades certificadoras confiáveis são responsáveis por vincular chaves públicas às identidades de seus respectivos proprietários. Embora eficazes, essas infraestruturas introduzem maior complexidade, custos operacionais e dependência de entidades centrais \cite{rfc_tls}.

Em função dessas limitações, sistemas de comunicação modernos adotam uma abordagem híbrida, na qual mecanismos assimétricos são empregados nas fases iniciais de autenticação e estabelecimento de chaves, enquanto algoritmos simétricos são utilizados para a proteção eficiente dos dados transmitidos. Nesse contexto, observa-se que, embora a criptografia forneça mecanismos consolidados para a proteção da informação, o processo de estabelecimento seguro de chaves permanece como um elemento crítico dos sistemas criptográficos tradicionais.

Essa constatação motiva a investigação de abordagens complementares para o estabelecimento de chaves secretas, especialmente em ambientes sem fio, conduzindo naturalmente ao estudo de técnicas baseadas em propriedades físicas do canal de comunicação.

\subsection{Alice, Bob e Eve: Modelo Geral de Segurança}

A análise de mecanismos de segurança em sistemas de comunicação sem fio é tradicionalmente conduzida a partir de um modelo conceitual composto por três entidades fundamentais, denominadas Alice, Bob e Eve. Nesse modelo geral de segurança, Alice e Bob representam dispositivos legítimos pertencentes à rede que desejam estabelecer entre si uma chave criptográfica simétrica para posterior comunicação segura, enquanto Eve representa um invasor passivo, cuja intenção é observar o meio de transmissão e extrair informações confidenciais sem interferir diretamente no processo de comunicação \cite{wyner_wiretap, bloch_wireless_security}.

No contexto deste trabalho, considera-se um cenário de comunicações móveis no qual Alice e Bob não se comunicam diretamente entre si. Ambos recebem sinais provenientes de uma estação rádio base ou ponto de acesso, que atua como fonte comum de excitação do canal sem fio. A partir do recebimento desse sinal comum, Alice e Bob realizam a sondagem do canal de comunicação, explorando as propriedades físicas do meio para possibilitar o estabelecimento de uma chave criptográfica simétrica compartilhada. Esse tipo de arquitetura é característico de redes celulares modernas e de aplicações de Internet das Coisas (\ac{IoT}), nas quais múltiplos dispositivos compartilham a mesma infraestrutura de acesso \cite{zhang_pks_survey}.

A Figura~\ref{fig:modelo_sistema} ilustra o modelo geral de segurança considerado neste trabalho. A estação rádio base transmite um sinal $x$, que é recebido por Alice, Bob e pelo invasor Eve através de canais sem fio distintos. O sinal recebido por cada entidade pode ser descrito pelas expressões
\[
y_A = h_A x + n_A, \quad
y_B = h_B x + n_B, \quad
y_E = h_E x + n_E,
\]
em que $h_A$, $h_B$ e $h_E$ representam os coeficientes complexos de desvanecimento dos canais entre a estação rádio base e Alice, Bob e Eve, respectivamente, enquanto $n_A$, $n_B$ e $n_E$ correspondem aos termos de ruído aditivo nos receptores.

\begin{figure}[!t]
    \centering
    \includegraphics[width=0.85\linewidth]{figuras/Figura1.png}
    \caption{Modelo geral de segurança com estação rádio base, usuários legítimos e invasor passivo.}
    \label{fig:modelo_sistema}
\end{figure}

Devido à proximidade espacial entre Alice e Bob e às condições semelhantes de propagação, os coeficientes de canal $h_A$ e $h_B$ apresentam correlação estatística significativa dentro do tempo de coerência do canal. Como resultado, as amostras do canal observadas por Alice e Bob são estatisticamente correlacionadas, constituindo a principal fonte de entropia compartilhada explorada no processo de geração de chaves em camada física \cite{mathur_pks, zhang_pks_survey}.

Por outro lado, o invasor Eve encontra-se em uma posição espacial distinta e observa um canal caracterizado por um coeficiente $h_E$ estatisticamente decorrelacionado em relação aos canais legítimos. Consequentemente, as observações realizadas por Eve apresentam baixa correlação com aquelas obtidas por Alice e Bob, dificultando de forma significativa a reconstrução da chave criptográfica. Assume-se ainda que Eve possui acesso irrestrito ao canal público utilizado nas etapas de reconciliação da informação, caracterizando um invasor passivo de acordo com o modelo clássico de segurança da informação \cite{maurer_secret_key}.

Esse modelo geral estabelece o cenário de sistema adotado neste trabalho e fornece a base necessária para a descrição das propriedades estatísticas do canal sem fio e das etapas do processo de geração de chaves criptográficas em camada física, discutidas nas subseções seguintes.


\subsection{O Canal Sem Fio e Suas Propriedades}

A geração de chaves criptográficas em camada física fundamenta-se diretamente no comportamento estatístico do canal sem fio. Diferentemente de modelos clássicos de comunicação ponto a ponto, o sistema considerado neste trabalho assume que uma estação rádio base ou ponto de acesso transmite sinais que são recebidos de forma independente por Alice e Bob. Assim, os canais experimentados por cada dispositivo não são idênticos, mas apresentam correlação estatística devido à proximidade espacial e às condições semelhantes de propagação.

O canal sem fio é modelado como um canal com desvanecimento aleatório, no qual o sinal transmitido sofre variações de amplitude e fase ao longo do tempo e da frequência. Essas variações decorrem da propagação multipercurso, na qual múltiplas réplicas do sinal chegam ao receptor após reflexões, difrações e espalhamentos causados por objetos presentes no ambiente. Como consequência, o sinal recebido pode ser descrito como a superposição ponderada dessas componentes, resultando em flutuações aleatórias conhecidas como desvanecimento (\textit{fading}) \cite{tse_viswanath}.

A Figura~\ref{fig:Figura2.png} ilustra um cenário típico de propagação sem fio, evidenciando o efeito de multipercurso e a correlação espacial entre os canais observados pelos dispositivos legítimos. Observa-se que Alice e Bob encontram-se separados por uma distância inferior a uma fração do comprimento de onda, enquanto o invasor Eve está localizado a uma distância significativamente maior, condição associada à decorrelação espacial do canal.

\begin{figure}[!t]
    \centering
    \includegraphics[width=0.85\linewidth]{figuras/Figura2.png}
    \caption{Cenário de propagação evidenciando o efeito de multipercurso e a correlação espacial entre os terminais legítimos.}
    \label{fig:cenario_propagacao}
\end{figure}


Matematicamente, o sinal recebido por um dispositivo pode ser expresso como
\begin{equation}
y = h x + n,
\end{equation}
em que $x$ representa o sinal transmitido pela estação rádio base ou ponto de acesso, $h$ é o coeficiente complexo de desvanecimento do canal sem fio e $n$ corresponde ao ruído aditivo no receptor, usualmente modelado como ruído branco Gaussiano aditivo (\ac{AWGN}) \cite{proakis_digital}.

Como Alice e Bob recebem sinais provenientes da mesma fonte transmissora e encontram-se em posições espacialmente próximas, os coeficientes de canal $h_A$ e $h_B$ apresentam correlação estatística significativa. Essa correlação constitui a principal fonte de entropia compartilhada explorada no processo de geração de chaves criptográficas em camada física \cite{mathur_pks, bloch_wireless_security}.

Por outro lado, um invasor localizado a uma distância superior a uma fração do comprimento de onda observa um canal caracterizado por coeficientes de desvanecimento estatisticamente distintos. Estudos clássicos demonstram que, nessas condições, as observações realizadas pelo invasor apresentam correlação significativamente reduzida em relação às observações dos usuários legítimos, limitando sua capacidade de inferir a informação compartilhada entre Alice e Bob \cite{mathur_pks, bloch_wireless_security, maurer_secret_key}.

Essas propriedades — desvanecimento aleatório, correlação estatística entre canais legítimos e decorrelação espacial em relação ao invasor — constituem a base física que viabiliza a extração de informação secreta diretamente do meio de propagação. Na subseção seguinte, descreve-se como as observações do canal são sistematicamente processadas no processo de geração de chaves criptográficas em camada física.


\subsection{Processo de Geração de Chaves em Camada Física}

O processo de geração de chaves criptográficas em camada física fundamenta-se na exploração das propriedades de correlação estatística do canal sem fio observadas por usuários legítimos e na independência estatística das observações realizadas por um invasor espacialmente separado. Diferentemente dos mecanismos criptográficos tradicionais, cuja segurança está associada à dificuldade computacional de certos problemas matemáticos, a abordagem em camada física explora características físicas do canal que limitam intrinsecamente a capacidade de um terceiro em obter informações correlacionadas com o segredo gerado.

De forma geral, o processo de geração de chaves pode ser decomposto em quatro etapas fundamentais: sondagem do canal, amostragem e quantização das observações, reconciliação da informação e amplificação de privacidade. Essas etapas visam transformar sinais analógicos contínuos, afetados pelo canal e pelo ruído do receptor, em uma chave binária compartilhada, altamente correlacionada entre Alice e Bob e com vazamento de informação estatisticamente limitado para um invasor \cite{mathur_pks, bloch_wireless_security}. A Figura~\ref{fig:Processo_Geração} ilustra essas etapas de forma esquemática.

\begin{figure}[!t]
\centering
\includegraphics[width=0.85\linewidth]{figuras/ProcessoGer.png}
\caption{Etapas fundamentais do processo de geração de chaves em camada física}
\label{fig:Processo_Geração}
\end{figure}


\subsubsection{Sondagem e Amostragem do Canal}

Na etapa de sondagem do canal, Alice e Bob recebem sinais de referência transmitidos por meio do enlace sem fio, de modo que cada dispositivo observa um sinal contínuo no tempo afetado pelas características do canal de propagação e pelo ruído do receptor. Essas observações não correspondem diretamente a amostras do canal, mas sim a sinais recebidos cuja estatística reflete as propriedades físicas do meio sem fio.

Considerando um modelo de canal de desvanecimento plano e quase-estático, o sinal recebido pelo dispositivo $i \in \{A,B\}$ pode ser modelado como
\begin{equation}
y_i(t) = h_i(t)\,x(t) + n_i(t),
\end{equation}
em que $x(t)$ representa o sinal de sondagem transmitido pela infraestrutura de rede (por exemplo, um ponto de acesso ou estação rádio base), $h_i(t)$ é o coeficiente complexo do canal entre a infraestrutura e o dispositivo $i$, e $n_i(t)$ denota o ruído aditivo no receptor.

Em sistemas que operam em modo \ac{TDD}, a reciprocidade do canal pode ser assumida em nível físico, de modo que os coeficientes do canal direto e reverso sejam aproximadamente iguais quando estimados dentro do tempo de coerência do canal \cite{tse_viswanath}. Assim, tem-se
\begin{equation}
h_A(t) \approx h_B(t),
\end{equation}
sendo essa aproximação limitada por ruído térmico, erros de estimação e imperfeições de hardware, como desbalanceamentos entre as cadeias de transmissão e recepção.

A partir do sinal contínuo recebido durante a sondagem, cada dispositivo realiza a amostragem temporal dessas observações, obtendo sequências discretas que preservam a correlação estatística imposta pelo canal. Essas sequências podem ser representadas como
\begin{equation}
\mathbf{y}_A = \{y_A(1), y_A(2), \dots, y_A(N)\},
\end{equation}
\begin{equation}
\mathbf{y}_B = \{y_B(1), y_B(2), \dots, y_B(N)\}.
\end{equation}

Devido à reciprocidade do canal, as sequências $\mathbf{y}_A$ e $\mathbf{y}_B$ apresentam elevada correlação estatística. Em contraste, um invasor passivo localizado a uma distância superior a aproximadamente meio comprimento de onda observa sinais associados a coeficientes de canal estatisticamente independentes, em decorrência da decorrelação espacial do desvanecimento multipercurso \cite{mathur_pks}. Essa diferença estatística constitui a base para a geração de segredos compartilhados em camada física.


\subsubsection{Quantização}

O processo de quantização tem como objetivo converter o sinal recebido, já amostrado no domínio discreto do tempo, em uma representação discreta no domínio de símbolos binários ou $M$-ários, possibilitando sua utilização no processo de geração de chaves criptográficas em camada física.

Considere $r_i(k)$ como o sinal recebido e amostrado pelo nó $i \in \{A,B\}$ no instante $k$. A partir desse sinal, extrai-se uma métrica escalar $z_i(k)$, como a potência instantânea do sinal recebido ou a magnitude de sua envoltória, que será utilizada como entrada do processo de quantização.

De forma geral, a quantização pode ser representada por
\begin{equation}
b_i(k) = Q\big(z_i(k)\big),
\end{equation}
em que $Q(\cdot)$ denota a função de quantização responsável por mapear a métrica escalar $z_i(k)$ em símbolos discretos.

Um exemplo elementar é a quantização binária por limiar, definida como
\begin{equation}
b_i(k) =
\begin{cases}
1, & z_i(k) \geq \tau, \\
0, & z_i(k) < \tau,
\end{cases}
\end{equation}
em que $\tau$ representa um limiar previamente estabelecido. Estratégias mais elaboradas podem empregar múltiplos níveis de quantização, visando aumentar a taxa de geração de bits, ao custo de maior sensibilidade às imperfeições do canal e ao ruído \cite{bloch_wireless_security}.

Devido às assimetrias introduzidas pelo canal sem fio, pelo ruído aditivo e por imperfeições nos processos de estimação e amostragem, as sequências discretas geradas a partir dos sinais recebidos por Alice e Bob,
\begin{equation}
\mathbf{b}_A \neq \mathbf{b}_B,
\end{equation}
podem diferir em determinadas posições. No modelo de sistema considerado neste trabalho, tais discrepâncias são tratadas por meio de um processo estruturado de reconciliação da informação, baseado no uso de códigos de correção de erros do tipo BCH, descrito na subseção seguinte.


\subsubsection{Reconciliação da Informação}

A etapa de reconciliação da informação tem como objetivo eliminar as discrepâncias existentes entre as sequências discretas obtidas por Alice e Bob após o processo de quantização, assegurando que ambas as partes compartilhem uma sequência binária idêntica ao final dessa fase.

No modelo de sistema considerado neste trabalho, Alice dispõe de uma sequência binária
\begin{equation}
\mathbf{b}_A = \{b_A(1), b_A(2), \dots, b_A(N)\},
\end{equation}
enquanto Bob possui uma versão correlacionada
\begin{equation}
\mathbf{b}_B = \{b_B(1), b_B(2), \dots, b_B(N)\},
\end{equation}
obtida a partir da observação do mesmo sinal de sondagem sob diferentes realizações de ruído e imperfeições do receptor. A relação entre $\mathbf{b}_A$ e $\mathbf{b}_B$ pode ser modelada como um canal binário com erros.

Para corrigir essas discrepâncias, adota-se neste trabalho um processo de reconciliação baseado em códigos de correção de erros do tipo BCH. Esses códigos são escolhidos por apresentarem estrutura algébrica bem definida, capacidade de correção ajustável e complexidade computacional compatível com dispositivos com recursos limitados.

O princípio da reconciliação consiste na transmissão de \emph{informações auxiliares de paridade} por meio de um canal público, de modo que Bob possa corrigir sua sequência sem que a chave em si seja revelada. 

No contexto de códigos corretores de erros, a \emph{síndrome} corresponde a uma representação compacta das inconsistências entre uma palavra recebida e uma palavra válida do código, sendo obtida a partir da matriz de verificação de paridade e amplamente utilizada em processos de decodificação para identificação e correção de erros.

No contexto dos códigos BCH, essas informações auxiliares assumem a forma da síndrome, que corresponde a um conjunto de verificações de paridade calculadas a partir da sequência binária de Alice.

Seja um código BCH linear definido pelos parâmetros $(n,k,t)$, em que $n$ representa o comprimento da palavra-código, $k$ o comprimento da palavra de informação e $t$ a capacidade máxima de correção de erros. Denotando por $\mathbf{H}$ a matriz de verificação de paridade do código, Alice calcula a síndrome associada à sua sequência como
\begin{equation}
\mathbf{s} = \mathbf{H} \, \mathbf{b}_A^{\mathrm{T}},
\end{equation}
em que as operações são realizadas no corpo finito apropriado. A síndrome $\mathbf{s}$ não contém diretamente a sequência $\mathbf{b}_A$, mas fornece informações suficientes para identificar padrões de erro.

A síndrome é então transmitida a Bob por meio de um canal público. Utilizando essa informação auxiliar em conjunto com sua própria sequência $\mathbf{b}_B$, Bob executa o algoritmo de decodificação do código BCH para estimar o vetor de erro $\mathbf{e}$. A sequência reconciliada é obtida por
\begin{equation}
\hat{\mathbf{b}}_A = \mathbf{b}_B \oplus \mathbf{e},
\end{equation}
em que $\oplus$ denota a soma módulo dois. Caso o número de erros esteja dentro da capacidade de correção $t$ do código, a decodificação é bem-sucedida e tem-se $\hat{\mathbf{b}}_A = \mathbf{b}_A$.

Embora a síndrome não revele diretamente a sequência binária compartilhada, sua transmissão em um canal público introduz um vazamento controlado de informação ao potencial invasor. Esse vazamento reduz a entropia da chave condicionada às observações do adversário, tornando necessária a aplicação de uma etapa adicional de amplificação de privacidade, descrita na subseção seguinte \cite{maurer_secret_key}.



\subsubsection{Amplificação de Privacidade}

A amplificação de privacidade constitui a etapa final do processo de geração de chaves em camada física e tem como objetivo neutralizar qualquer informação parcial que um invasor possa ter obtido ao longo das fases anteriores, em especial durante a reconciliação da informação. Embora o invasor não tenha acesso direto às observações do canal legítimo, a troca de informações auxiliares em canal público inevitavelmente provoca vazamento de informação, reduzindo a entropia efetiva da sequência binária compartilhada entre Alice e Bob.

Nesse contexto, a amplificação de privacidade é fundamental para garantir que a chave final seja estatisticamente independente das observações e do conhecimento do invasor. Do ponto de vista teórico, essa etapa é sustentada pelos resultados clássicos de Maurer e Wolf, que demonstram ser possível extrair uma chave secreta segura a partir de uma sequência parcialmente comprometida por meio da aplicação de funções hash pertencentes a famílias universais, desde que a entropia residual da sequência seja suficiente \cite{maurer_secret_key}.

Seja $\mathbf{k} \in \{0,1\}^n$ a sequência binária comum obtida após a reconciliação da informação. A amplificação de privacidade consiste na aplicação de uma função hash criptograficamente segura que mapeia essa sequência em uma nova sequência de menor comprimento $m < n$, dada por
\begin{equation}
\mathbf{s} = H(\mathbf{k}),
\end{equation}
em que $H(\cdot)$ representa a função hash utilizada. A redução do comprimento da chave está diretamente relacionada à eliminação da informação potencialmente conhecida pelo invasor, de forma que a chave final apresente elevada entropia condicional mesmo na presença de vazamento prévio.

No modelo adotado neste trabalho, a amplificação de privacidade é implementada por meio do algoritmo SHA-256, pertencente à família SHA-2, amplamente padronizada e recomendada por órgãos de normalização como o \textit{National Institute of Standards and Technology} (NIST) \cite{nist_sha}. A escolha dessa função justifica-se pelo fato de que pequenas variações na sequência de entrada resultam em saídas completamente distintas, característica essencial para impedir que informações residuais presentes na sequência reconciliada sejam refletidas na chave final.

Além disso, a estrutura criptográfica do SHA-256 dificulta computacionalmente tanto a obtenção da sequência original a partir da chave final quanto a construção de sequências distintas que produzam a mesma saída. Essas propriedades asseguram que um invasor, mesmo conhecendo a função hash utilizada e possuindo acesso às informações públicas trocadas durante a reconciliação, não consiga inferir a sequência binária original nem produzir chaves equivalentes à chave legítima compartilhada.

Do ponto de vista prático, o SHA-256 apresenta ainda elevada eficiência computacional, o que o torna adequado para implementação em dispositivos com restrições de processamento e energia, como terminais móveis e dispositivos da Internet das Coisas. Dessa forma, sua utilização na etapa de amplificação de privacidade permite reforçar significativamente a segurança da chave final sem introduzir sobrecarga relevante ao sistema.

Ao final desse processo, Alice e Bob compartilham uma chave secreta estatisticamente segura, cuja confidencialidade está fundamentada nas propriedades físicas do canal sem fio e reforçada por mecanismos criptográficos de processamento da informação. Essa chave pode então ser utilizada diretamente em algoritmos criptográficos simétricos convencionais, como o AES, atendendo aos requisitos de segurança do sistema considerado.



