\documentclass[10pt,conference]{inatel} 
%Use esse arquivo para incluir novos pacotes

\usepackage[%usado para determinar medidas
top=1.78cm,
bottom=1.78cm,
left=1.65cm,
right=1.65cm,
headsep=0cm,
%showframe
]{geometry}
%\usepackage[justification=centering]{caption}
\usepackage{times}
\usepackage{enumitem}%redefinir espacos itemize
\usepackage{graphicx}
\usepackage{url,hyperref}
\usepackage[utf8]{inputenc}
\usepackage{float}%mais controle para manipular figuras
\usepackage{caption}%manipular legenda da figura e tabela
\usepackage{mathtools}%equacoes
\usepackage[hang,flushmargin]{footmisc} 
\usepackage{xcolor}
\usepackage{wrapfig} %usado para envolver figura com texto
%\usepackage[portuguese]{babel}
\usepackage{fancyhdr}%criacao do cabecalho
\usepackage{etoolbox}
\usepackage[export]{adjustbox}%mais controle para ajustar tamanho da tabela
\usepackage{multirow}%usado para mesclar linhas em tabelas
\usepackage{comment}%ambiente para comentario
\usepackage{relsize} %usado por comandos \mathlarger
\usepackage{subfigure}
\usepackage{acro}


%Referencia bibliografica
\usepackage[
    style=numeric,
    sorting=none,
    maxbibnames=10]{biblatex}
\addbibresource{referencia.bib}

%Idioma. Use "english" para trabalhos em inglês
\usepackage[brazil]{babel}

%Inserindo Códigos
\usepackage{listings}
\definecolor{dkgreen}{rgb}{0,0.6,0}
\definecolor{gray}{rgb}{0.5,0.5,0.5}
\definecolor{mauve}{rgb}{0.58,0,0.82}
%\lstset{
%  language=Python,                
%  basicstyle=\footnotesize,           
%   numbers=false,                   
%   numberstyle=\tiny\color{gray},  
%   stepnumber=2,                             
%   numbersep=5pt,                  
%   backgroundcolor=\color{white},    
%   showspaces=false,               
%   showstringspaces=false,         
%   showtabs=false,                 
%   frame=single,                   
%   rulecolor=\color{black},        
%   tabsize=2,                      
%   captionpos=b,                   
%   breaklines=true,                
%   breakatwhitespace=false,        
%   title=\lstname,                               
%   keywordstyle=\color{blue},          
%   commentstyle=\color{dkgreen},       
%   stringstyle=\color{mauve}    
% }
\input{configuracoes/config.tex}
\input{configuracoes/acros}
\usepackage[utf8]{inputenc}
\usepackage{afterpage}
\usepackage{amsmath,amssymb,amsfonts,mathrsfs,physics,bm,amsthm}
\usepackage{algorithmic}
\usepackage{graphicx}
\usepackage{tasks}
%\usepackage[resetlabels]{multibib}
\usepackage{textcomp}
\usepackage{xcolor}
\usepackage{wrapfig}
\usepackage{caption}
\usepackage{float}
\usepackage{morewrites}
\usepackage{acro}
\usepackage{mathtools}
%\usepackage{cite}
\DeclarePairedDelimiter\ceil{\lceil}{\rceil}
\DeclarePairedDelimiter\floor{\lfloor}{\rfloor}
\usepackage{textcomp}
\newtheorem{proposition}{Proposition}
\usepackage{makecell, cellspace, caption}
\usepackage{array}
\hyphenation{op-tical net-works semi-conduc-tor IEEE-Xplore co-mmer-cia-li-za-tion}

\begin{document}

\title{Segurança em Camada Física: Estabelecimento de Chaves Criptográficas para Comunicações Móveis de Próxima Geração}
\author
{
  \IEEEauthorblockN{Pedro Henrique D. Frugoli, Henrique R. Mendonça, Guilherme P. Aquino \\ Luciano L. Mendes e Vanessa M. Rennó}
  \IEEEauthorblockA{Centro de Referência em Radiocomunicações (CRR), Laboratório de Cyber Segurança e Internet das Coisas (CS\&I Lab.) \\ Instituto Nacional de Telecomunicações - Inatel\\
               pedro.frugoli@ges.inatel.br, henrique.rodrigues@ges.inatel.br, \\  guilhermeaquino@inatel.br, lucianol@inatel.br, vanessarenno@inatel.br}
}

\maketitle

%
\begin{resumo}
Este trabalho apresenta a implementação e validação experimental de um sistema de geração de chaves criptográficas em camada física para redes 5G e Internet das Coisas, explorando correlação espacial de canais sem fio observados por dispositivos legítimos conectados à mesma estação base. O sistema integra modulação BPSK/QPSK, protocolo de reconciliação \textit{code-offset} com códigos BCH(127,64,10) e amplificação de privacidade via SHA-256. Sete experimentos sistemáticos demonstram: (i) SNR mínimo de 13--15~dB garante taxa de discordância de chaves (KDR) nula entre usuários legítimos; (ii) complexidade computacional de 0.489~ms para codificação e decodificação BCH viabiliza aplicação em dispositivos IoT de baixo custo; (iii) alta capacidade de correção do código BCH (t=10) permite KDR inferior a 0.03\% sem necessidade de limiarização adaptativa complexa, maximizando a eficiência de geração de chaves; (iv) descorrelação espacial garante segurança contra espionagem passiva a partir de 20~cm de separação (BER Eve $\approx$ 50\%); (v) validação em cinco perfis IoT distintos---sensores estáticos, \textit{wearables}, veículos urbanos (60~km/h), drones (40~km/h) e NB-IoT---revela que o sistema opera mesmo em cenários de alta mobilidade com correlação temporal reduzida ($\rho=0.16$), indicando que erro de estimação controlado é fator mais crítico que correlação temporal para viabilidade prática. Os resultados demonstram alternativa energeticamente eficiente aos métodos criptográficos convencionais para estabelecimento de chaves em redes de próxima geração.
\end{resumo}

\begin{palavraschave}
Geração de chaves em camada física, segurança 5G/IoT, códigos BCH, reconciliação \textit{code-offset}, correlação espacial.
\end{palavraschave}

\begin{abstract}
This work presents the implementation and experimental validation of a physical-layer key generation system for 5G and Internet of Things networks, exploiting spatial correlation of wireless channels observed by legitimate devices connected to the same base station. The system integrates BPSK/QPSK modulation, \textit{code-offset} reconciliation protocol with BCH(127,64,10) codes, and privacy amplification via SHA-256. Seven systematic experiments demonstrate: (i) minimum SNR of 13--15~dB ensures zero key disagreement rate (KDR) between legitimate users; (ii) computational complexity of 0.489~ms for BCH encoding and decoding enables deployment on low-cost IoT devices; (iii) high error-correction capability of BCH code (t=10) achieves KDR below 0.03\% without requiring complex adaptive thresholding, maximizing key generation efficiency; (iv) spatial decorrelation ensures security against passive eavesdropping beyond 20~cm separation (Eve's BER $\approx$ 50\%); (v) validation across five distinct IoT profiles---static sensors, wearables, urban vehicles (60~km/h), drones (40~km/h), and NB-IoT---reveals that the system operates even in high-mobility scenarios with reduced temporal correlation ($\rho=0.16$), indicating that controlled estimation error is more critical than temporal correlation for practical viability. Results demonstrate an energy-efficient alternative to conventional cryptographic methods for key establishment in next-generation networks.
\end{abstract}

\begin{IEEEkeywords}
Physical-layer key generation, 5G/IoT security, BCH codes, code-offset reconciliation, spatial correlation.
\end{IEEEkeywords}

\section{Introdução}

A evolução das comunicações móveis, impulsionada pelo desenvolvimento das redes \ac{5G} e pelas pesquisas em \ac{6G}, tem intensificado a demanda por mecanismos de segurança mais eficientes e adaptados ao ambiente sem fio. Com o crescimento do número de dispositivos conectados e da troca de informações sensíveis, garantir a segurança dos dados transmitidos tornou-se um desafio central para pesquisadores e engenheiros de telecomunicações.

Uma comunicação segura deve atender a cinco pilares fundamentais da Segurança da Informação. A confidencialidade garante que apenas as partes autorizadas possam ter acesso ao conteúdo transmitido, impedindo a interceptação e o vazamento de dados. A integridade assegura que a informação não seja alterada de forma indevida durante o armazenamento ou transmissão, preservando seu conteúdo original. A autenticidade assegura que as entidades envolvidas na comunicação são genuínas, evitando ataques de falsificação e impersonificação. A irretratabilidade (ou não-repúdio) impede que uma das partes envolvidas neguem posteriormente ter participado de uma comunicação ou transação. Por fim, a disponibilidade garante que os dados e serviços estejam acessíveis sempre que necessário, evitando interrupções causadas por falhas ou ataques.

Dentre esses pilares, a confidencialidade desempenha um papel central na proteção das informações trafegadas. Sua implementação tradicional é realizada por algoritmos criptográficos, como \ac{AES} e \ac{RSA}, que dependem diretamente do estabelecimento seguro de chaves criptográficas simétricas ou assimétricas. No entanto, os métodos convencionais de troca de chaves — como Diffie–Hellman e infraestruturas de chave pública — envolvem elevado custo computacional e apresentam vulnerabilidades crescentes diante do avanço da computação quântica.

Esse cenário reforça a necessidade de mecanismos complementares para o estabelecimento de segredos, especialmente em dispositivos móveis e ambientes com capacidade computacional e energética limitadas. Entre as alternativas emergentes, destaca-se a \ac{PLS}, abordagem que explora propriedades físicas intrínsecas do canal de comunicação sem fio — como reciprocidade do canal, variações temporais e ruído térmico — para estabelecer chaves simétricas em ambos os dispositivos sem depender de transmissão explícita de chaves, reduzindo o custo computacional e aumentando a resiliência contra ataques.

Dentro desse contexto, o \ac{PKG} emerge como uma técnica promissora que permite a dois dispositivos legítimos estabelecerem uma chave criptográfica comum explorando as características aleatórias e recíprocas do canal de comunicação. Diferentemente dos métodos tradicionais que exigem infraestrutura de chave pública ou pré-distribuição de segredos, o \ac{PKG} utiliza as flutuações naturais do canal sem fio para gerar bits de entropia compartilhados entre transmissor e receptor, desde que ambos realizem medições do canal em um curto intervalo de tempo, respeitando o tempo de coerência do canal, e que o ambiente apresente variações suficientes para gerar entropia. Esta abordagem é particularmente relevante para redes \ac{5G}/\ac{6G} e sistemas \ac{IoT}, onde a escalabilidade, o baixo consumo energético e a maior resiliência contra ameaças quânticas são requisitos críticos. Em redes densas com milhares de dispositivos conectados simultaneamente, mecanismos que não dependem de infraestrutura centralizada de gerenciamento de chaves tornam-se essenciais para garantir a viabilidade operacional. Adicionalmente, dispositivos \ac{IoT} frequentemente operam com baterias limitadas, tornando inviável o uso intensivo de algoritmos criptográficos pesados. O \ac{PKG}, ao explorar o canal físico já utilizado para comunicação, oferece uma solução energeticamente eficiente. Por fim, com o avanço dos computadores quânticos, algoritmos tradicionais como RSA e Diffie–Hellman tornam-se vulneráveis, enquanto o \ac{PKG}, ao não depender de problemas matemáticos computacionalmente complexos, apresenta maior resiliência a esses ataques.

O presente trabalho apresenta a implementação e análise de um sistema completo de comunicação móvel que objetiva \ac{PKG}, investigando sua viabilidade prática e desempenho sob diferentes condições de canal e parâmetros de comunicação, tais como relação sinal-ruído (SNR), parâmetros de desvanecimento (sigma), modulação digital (BPSK/QPSK), e protocolo de reconciliação code-offset com suporte de códigos corretores de erro BCH. Os resultados experimentais demonstram, de forma geral, um bom estabelecimento de chaves criptográficas sob diversas condições de operação.

O restante deste artigo está organizado da seguinte forma: a Seção II apresenta a fundamentação teórica sobre \ac{PKG} e seus componentes principais; a Seção III descreve o modelo de sistema proposto; a Seção IV detalha a implementação e os parâmetros de simulação; e a Seção V apresenta os resultados obtidos e conclusões.

\section{Fundamentação Teórica}

A segurança da informação em sistemas de comunicação tem sido tradicionalmente garantida por mecanismos de criptografia implementados nas camadas superiores da pilha de protocolos. Os pilares da confidencialidade, integridade, autenticidade, irretratabilidade e disponibilidade dependem, em geral, de algoritmos criptográficos robustos e de uma gestão eficiente de chaves simétricas ou assimétricas. Entretanto, o estabelecimento e a distribuição de chaves seguras permanecem como desafios centrais, especialmente em redes móveis altamente dinâmicas e em ambientes IoT com recursos computacionais limitados.

\subsection{Criptografia Simétrica e Assimétrica}

A criptografia moderna é baseada principalmente em dois tipos de chaves. A criptografia assimétrica (ou de chave pública) utiliza um par de chaves matematicamente relacionadas — uma chave pública e uma chave privada — como ocorre nos algoritmos RSA, ECC e variantes de Diffie–Hellman. Embora ofereçam flexibilidade e segurança, esses métodos são computacionalmente custosos e, em muitos cenários, vulneráveis a ataques de computadores quânticos. Já a criptografia simétrica utiliza a mesma chave para cifrar e decifrar dados, como no caso do AES, sendo extremamente eficiente e amplamente usada em sistemas reais. No entanto, essa abordagem depende de que as partes envolvidas já possuam uma chave secreta compartilhada, o que nos leva ao problema da distribuição de chaves.

O sistema que desenvolvemos neste trabalho é baseado em criptografia simétrica. Ou seja, o objetivo da geração de chaves em camada física é justamente criar, de forma automática e segura, uma chave secreta que será posteriormente utilizada por algoritmos como AES.

\subsection{Alice, Bob e Eve: Modelo Clássico de Segurança}

Para explicar cenários de segurança em comunicações, utiliza-se uma convenção simples: Alice é o transmissor legítimo, Bob é o receptor legítimo e Eve (de \textit{eavesdropper}) é a atacante que tenta interceptar a comunicação. Esses nomes não possuem significado técnico; são apenas uma forma padronizada de ilustrar interações. O objetivo é fazer com que Alice e Bob cheguem a uma chave secreta idêntica, enquanto Eve, por estar fora da região física de reciprocidade, não consiga obter a mesma informação.

\subsection{O Canal Sem Fio e Suas Propriedades}

Para entender a geração de chaves físicas, é necessário compreender como funciona a transmissão sem fio. Quando Alice transmite um sinal para Bob, esse sinal passa por um meio físico sujeito a fenômenos aleatórios como desvanecimento (\textit{fading}), que corresponde a oscilações naturais na potência do sinal causadas por múltiplos caminhos, reflexões e obstáculos; ruído térmico, proveniente de interferências aleatórias causadas pelo movimento de elétrons nos circuitos eletrônicos; e variação temporal, decorrente de mudanças no canal ao longo do tempo devido ao movimento dos usuários ou do ambiente.

Além disso, o canal sem fio exibe duas propriedades fundamentais para a geração de chaves. A primeira é a reciprocidade: dentro do tempo de coerência, o canal Alice–Bob é estatisticamente igual ao canal Bob–Alice. A segunda é a decorrelação espacial: pequenas mudanças na posição — tipicamente acima de \(\lambda/2\) — fazem com que o canal observado por Eve seja completamente diferente. Essas características tornam possível gerar segredos compartilhados diretamente do canal sem fio, sem transmissão explícita de chaves.

\subsection{Processo de Geração de Chaves em Camada Física}

A geração de chaves em camada física ocorre por meio de quatro etapas principais: amostragem, quantização, reconciliação e amplificação de privacidade. A seguir, descrevemos cada etapa de maneira detalhada.

\subsubsection{1. Amostragem do Canal}

Alice transmite um sinal piloto (conhecido por Bob) e Bob estima o ganho do canal. Em seguida, Bob transmite outro sinal piloto e Alice estima o canal novamente. Devido à reciprocidade, ambas as estimativas são muito parecidas. O ganho do canal pode ser representado, por exemplo, como uma série temporal:

\[
h(t) = h_\text{real}(t) + j h_\text{imag}(t)
\]

No nosso caso, utilizamos o módulo do ganho como fonte de aleatoriedade.

\subsubsection{2. Quantização}

Para transformar as amostras contínuas do canal em bits, é necessário aplicar um procedimento de quantização. Isso significa decidir, com base em limites específicos, se cada amostra será representada como:

\[
0 \quad \text{ou} \quad 1
\]

Esse processo gera duas sequências binárias:

\[
K_A \quad \text{(chave inicial de Alice)} \\
K_B \quad \text{(chave inicial de Bob)}
\]

Como há ruído e pequenas discrepâncias no canal, essas sequências não são perfeitamente iguais, surgindo então a necessidade de reconciliá-las.

\subsubsection{3. Reconciliação de Chaves}

A reconciliação tem como objetivo reduzir o \textit{Key Disagreement Rate} (KDR), corrigindo os bits que diferem entre \(K_A\) e \(K_B\). Para isso, utiliza-se um código de correção de erros — neste trabalho, um código BCH. A reconciliação funciona da seguinte forma: Bob escolhe uma sequência aleatória \(C\), chamada código. Ele combina \(C\) com sua chave inicial por meio de XOR ou soma módulo-2, produzindo a chamada síndrome pública. Essa síndrome é enviada pelo canal público para Alice. Como Alice possui \(K_A\), ela utiliza a síndrome e o decodificador BCH para recuperar \(C\). Uma vez que recupera \(C\), ela produz a mesma chave corrigida que Bob.

O ponto crucial é que Eve, que não possui \(K_A\) nem \(K_B\), não consegue recuperar \(C\) apenas com a síndrome, pois isso exigiria conhecimento prévio da estrutura de erros entre as sequências legítimas.

\subsubsection{4. Amplificação de Privacidade}

Mesmo após a reconciliação, a síndrome pública pode ter revelado pequenas quantidades de informação sobre a chave. Para remover qualquer possível vazamento, aplica-se uma função de hashing criptográfico, como SHA-256, reduzindo a chave a uma versão final curta, porém completamente segura:

\[
K_\text{final} = \text{Hash}(C)
\]

Esse processo garante que, mesmo com acesso às informações públicas trocadas durante a reconciliação, Eve não obtenha vantagem alguma.

\subsection{Key Disagreement Rate (KDR)}

O \textit{Key Disagreement Rate} é a proporção de bits diferentes entre \(K_A\) e \(K_B\). Ele representa uma métrica essencial para avaliar a viabilidade prática do método. Sabe-se que a proximidade física entre Alice e Bob reduz significativamente o KDR, enquanto o distanciamento tende a degradar a reciprocidade.

\subsection{Resumo da Fundamentação}

Com essas explicações, concluímos que a geração de chaves em camada física é fundamentada em propriedades estatísticas do canal sem fio, como reciprocidade, decorrelação espacial e desvanecimento; no uso de quantização para extrair bits de entropia; em técnicas de reconciliação baseadas em códigos BCH para corrigir discrepâncias; na amplificação de privacidade para garantir confidencialidade absoluta; e em métricas como o KDR para avaliar a qualidade do método. Esses elementos servem de base para o modelo de simulação implementado neste trabalho.

\section{Materials and Methods}
\label{sec:metodos}

Esta seção apresenta detalhadamente os métodos utilizados para a implementação,
simulação e avaliação do sistema de geração de chaves em camada física.
Descrevem-se o ambiente de desenvolvimento, a modelagem do canal, o processo de
extração de aleatoriedade, os procedimentos de reconciliação e amplificação de
privacidade, bem como os experimentos conduzidos para análise do desempenho.

\subsection{Descrição Geral da Simulação}

O objetivo da simulação é reproduzir matematicamente e computacionalmente um
sistema de geração de chaves baseado na reciprocidade do canal em comunicações
sem fio. O sistema modela dois nós legítimos, Alice e Bob, trocando sinais
e obtendo estimativas do canal em momentos próximos o suficiente para
garantir a reciprocidade. 

A simulação contempla:

\begin{itemize}
    \item geração de amostras de canal Rayleigh complexas;
    \item transmissão de bits através do canal com ruído \ac{AWGN};
    \item modulação \ac{BPSK} ou \ac{QPSK} com detecção coerente;
    \item uso de códigos \ac{BCH} para reconciliação;
    \item uso de \ac{SHA}-256 para amplificação de privacidade;
    \item cálculo de métricas de desempenho (\ac{KDR}, concordância).
\end{itemize}

O pipeline completo permite avaliar, de forma controlada, como alterações no
cenário físico afetam a concordância das chaves e a segurança estatística do
sistema.

\subsection{Ambiente de Desenvolvimento}

\subsubsection{Linguagem e Bibliotecas Utilizadas}

As simulações foram implementadas integralmente em Python 3.11 devido à sua
flexibilidade e à ampla disponibilidade de bibliotecas científicas. As
principais dependências incluem:

\begin{itemize}
    \item \textbf{NumPy}: geração de números complexos, manipulação vetorial e operações matriciais;
    \item \textbf{SciPy}: funções estatísticas e transformações auxiliares;
    \item \textbf{Matplotlib}: geração de gráficos de desempenho;
    \item \textbf{Galois}: implementação de códigos \ac{BCH} para reconciliação;
    \item \textbf{hashlib}: implementação do \ac{SHA}-256 para amplificação de privacidade;
    \item \textbf{tqdm}: barras de progresso para simulações longas.
\end{itemize}

Todas as simulações foram executadas em ambiente local, num computador com
processador {}, garantindo execução consistente e
reprodutível.

\subsubsection{Parâmetros de Simulação}

Os parâmetros controláveis do ambiente de simulação incluem:

\begin{itemize}
    \item número de bits transmitidos: tipicamente \(N = 127\) (comprimento do código \ac{BCH});
    \item níveis de \ac{SNR}: \(\mathrm{SNR} \in [0, 30] \, \mathrm{dB}\);
    \item parâmetro de escala Rayleigh: \(\sigma \in [0.5, 2.0]\);
    \item código \ac{BCH}: parâmetros \((n,k,t)\), com \(n=127\), \(k=64\), \(t=10\);
    \item tipo de modulação: \ac{BPSK} ou \ac{QPSK};
    \item variância do ruído \ac{AWGN}: calculada a partir da \ac{SNR} desejada.
\end{itemize}

Esses parâmetros foram escolhidos de forma a representar condições realistas e
cobrir uma ampla faixa de cenários físicos.

\subsection{Estrutura do Código}

\subsubsection{Módulos Principais}

A implementação foi organizada em módulos independentes, visando clareza e
modularidade:

\begin{itemize}
    \item \texttt{canal/canal.py} — simulação de canal Rayleigh com ruído \ac{AWGN} e modulação \ac{BPSK}/\ac{QPSK};
    \item \texttt{canal/modulacao.py} — modulação temporal realista com frequência de portadora;
    \item \texttt{codigos\_corretores/bch.py} — codificação e decodificação \ac{BCH};
    \item \texttt{pilares/reconciliacao.py} — reconciliação de chaves usando code-offset;
    \item \texttt{pilares/amplificacao\_privacidade.py} — aplicação do \ac{SHA}-256 para obtenção da chave final;
    \item \texttt{visualizacao/plotkdr.py} — geração de gráficos de métricas e desempenho;
    \item \texttt{util/binario\_util.py} — operações auxiliares com sequências binárias.
\end{itemize}

\subsubsection{Fluxo Geral de Execução}

O pipeline da simulação segue:

\begin{enumerate}
    \item geração de sequência de bits aleatória para Alice;
    \item geração do ganho de canal \(h \sim \mathcal{CN}(0,1)\) (Rayleigh);
    \item transmissão de Alice para Bob: \(y_B = h \cdot x_A + n_B\);
    \item transmissão de Bob para Alice: \(y_A = h \cdot x_B + n_A\) (reciprocidade);
    \item demodulação e detecção dos bits recebidos;
    \item reconciliação das sequências via \ac{BCH} usando code-offset;
    \item amplificação de privacidade via \ac{SHA}-256;
    \item cálculo das métricas (\ac{KDR} pré e pós-reconciliação, \ac{KDR} pós-amplificação).
\end{enumerate}

O fluxo permite avaliar precisamente o impacto da \ac{SNR}, distância e número de
amostras sobre a qualidade das chaves.

\subsection{Modelagem do Canal}

\subsubsection{Canal Rayleigh}

O canal é modelado como uma variável aleatória complexa:

\begin{equation}
h \sim \mathcal{CN}(0,1),
\end{equation}

resultando em módulo Rayleigh e fase uniforme. Esse modelo captura condições de
multipercurso sem linha de visada (\ac{NLoS}), típico de ambientes internos.

\subsubsection{Modelo de Ruído, Distância e Amostras}

Cada nó observa:

\begin{equation}
y = h x + n,
\end{equation}

onde \(x\) é o piloto e \(n \sim \mathcal{CN}(0,\sigma^2)\) representa \ac{AWGN}.

A influência da distância é incorporada por perda de percurso:

\begin{equation}
h_d = \frac{h}{d^\alpha},
\end{equation}

com \(\alpha \in [2,4]\) dependendo do ambiente.

Batches de amostras são gerados para cada estimativa, simula

%% Seção 05 - Seção IV

\section{Resultados e Discussão}
\label{sec:results}

Esta seção apresenta os resultados dos sete experimentos sistemáticos conduzidos para validação do sistema de geração de chaves em camada física. Os experimentos cobrem aspectos de desempenho, segurança, complexidade computacional e aplicabilidade prática em diferentes cenários \ac{IoT}. Todas as simulações empregaram análise de Monte Carlo com 1000 realizações independentes, garantindo significância estatística dos resultados.

\subsection{Experimento 1: Impacto da Relação Sinal-Ruído}

O primeiro experimento investiga o impacto da \ac{SNR} sobre a \ac{BMR} inicial e o \ac{KDR} após reconciliação. A \ac{SNR} foi variada logaritmicamente de $-10$~dB a $30$~dB em 18 pontos, mantendo fixos a correlação espacial $\rho = 0.9$, a modulação \ac{BPSK}, e o código \ac{BCH}(127,64,10).

A Figura~\ref{fig:exp01_snr} apresenta as curvas de \ac{BMR} e \ac{KDR} em função da \ac{SNR}, demonstrando o impacto da reconciliação \ac{BCH}. A Figura~\ref{fig:exp01_bmr} mostra que o \ac{BMR} decresce continuamente com a \ac{SNR}, comportamento esperado uma vez que o aumento da relação sinal-ruído reduz diretamente a probabilidade de erro na detecção dos símbolos transmitidos. Por sua vez, a Figura~\ref{fig:exp01_kdr} evidencia que o \ac{KDR} apresenta decaimento exponencial mais acentuado, atingindo valores nulos a partir de $13$--$15$~dB. Esse comportamento reflete a eficácia do código \ac{BCH}(127,64,10), cuja capacidade de correção de até $t=10$ erros por bloco permite reconciliar completamente as sequências quando o \ac{BMR} inicial é suficientemente baixo. 

Observa-se um fenômeno importante: em regiões de \ac{SNR} muito baixa (tipicamente abaixo de $0$~dB), podem ocorrer casos em que o \ac{KDR} após reconciliação supera ligeiramente o \ac{BMR} inicial. Esse comportamento aparentemente contraditório decorre de falhas de decodificação \ac{BCH}: quando o número de erros por bloco de 127 bits excede a capacidade de correção do código ($>10$ erros), o decodificador pode produzir uma palavra-código incorreta que diverge ainda mais da sequência original, aumentando temporariamente a discrepância. Esse efeito, documentado em trabalhos anteriores sobre códigos lineares~\cite{proakis_digital}, evidencia que o sistema opera adequadamente apenas acima de um \ac{SNR} mínimo onde a taxa de erro inicial é compatível com a capacidade corretiva do código empregado. Para o código \ac{BCH}(127,64,10), esse limiar situa-se em aproximadamente $8$--$9$~dB, como confirmado pela Tabela~\ref{tab:exp01_snr}. A Tabela~\ref{tab:exp01_snr} apresenta os valores numéricos obtidos para pontos selecionados.

\begin{figure*}[t]
\centering
\subfigure[BMR antes da reconciliação]{\includegraphics[width=0.48\textwidth]{figuras/exp01_variacao_snr_01_20260212_021057.png}\label{fig:exp01_bmr}}
\hfill
\subfigure[KDR após reconciliação BCH(127,64,10)]{\includegraphics[width=0.48\textwidth]{figuras/exp01_variacao_snr_02_20260212_021057.png}\label{fig:exp01_kdr}}
\caption{Impacto da SNR no desempenho do sistema de geração de chaves em camada física.}
\label{fig:exp01_snr}
\end{figure*}

\begin{table}[htbp]
\centering
\caption{Desempenho do sistema para valores selecionados de SNR.}
\label{tab:exp01_snr}
\begin{tabular}{cccc}
\hline
\textbf{SNR (dB)} & \textbf{BMR (\%)} & \textbf{KDR (\%)} & \textbf{Status} \\
\hline
$-10.00$ & $44.45$ & $47.49$ & Canal ruidoso \\
$0.00$ & $24.56$ & $37.12$ & Insuficiente \\
$8.82$ & $5.40$ & $2.79$ & Limiar de operação \\
$11.18$ & $3.38$ & $0.03$ & Operacional \\
$13.53$ & $2.06$ & $0.00$ & Ideal \\
$\geq 15.88$ & $\leq 1.27$ & $0.00$ & Ideal \\
\hline
\end{tabular}
\end{table}

Os resultados demonstram que um \ac{SNR} mínimo de $13$--$15$~dB é necessário para operação confiável do sistema, caracterizada por \ac{KDR} nula entre Alice e Bob. Esse requisito é plenamente alcançável em aplicações práticas de \ac{IoT}, sendo compatível com enlaces de comunicação típicos em redes celulares \ac{5G} e \ac{NB-IoT} em ambientes internos e urbanos~\cite{3gpp_nr_coverage, nb_iot_link_budget}, nos quais valores de \ac{SNR} superiores a 15~dB são rotineiramente observados para dispositivos em condições normais de operação.

\subsection{Experimento 2: Comparação entre BPSK e QPSK}

O segundo experimento compara o desempenho dos esquemas de modulação \ac{BPSK} (1~bit/símbolo) e \ac{QPSK} (2~bits/símbolo). Manteve-se correlação espacial $\rho = 0.9$ e variou-se a \ac{SNR} conforme o Experimento~1.

A Tabela~\ref{tab:exp02_modulacao} apresenta a comparação entre as duas modulações para valores selecionados de \ac{SNR}. Os resultados indicam desempenho muito similar entre \ac{BPSK} e \ac{QPSK}, com diferenças marginais de \ac{BMR} e \ac{KDR} que se tornam desprezíveis em regime de alta \ac{SNR}. A equivalência de desempenho é explicada pelo fato de que, embora \ac{QPSK} possua quatro símbolos e portanto maior probabilidade de erro de símbolo, a quantização binária aplicada (mapeamento para $\pm 1$ nas componentes I e Q) resulta em taxas de erro de bit similares para as duas modulações quando operando na mesma \ac{SNR}. Isso demonstra que a segurança e eficiência do sistema de geração de chaves não é significativamente afetada pela escolha entre \ac{BPSK} e \ac{QPSK}, permitindo que outros critérios (eficiência espectral, simplicidade de implementação) guiem essa decisão em aplicações práticas.

\begin{table}[htbp]
\centering
\small
\caption{Comparação de desempenho entre modulação BPSK e QPSK.}
\label{tab:exp02_modulacao}
\begin{tabular}{c|cc|cc}
\hline
\multirow{2}{*}{\textbf{SNR (dB)}} & \multicolumn{2}{c|}{\textbf{BMR (\%)}} & \multicolumn{2}{c}{\textbf{KDR (\%)}} \\
\cline{2-5}
 & \textbf{BPSK} & \textbf{QPSK} & \textbf{BPSK} & \textbf{QPSK} \\
\hline
$8.82$ & $5.55$ & $5.47$ & $2.93$ & $3.37$ \\
$11.18$ & $3.40$ & $3.33$ & $0.15$ & $0.20$ \\
$13.53$ & $2.00$ & $2.03$ & $0.00$ & $0.03$ \\
$30.00$ & $0.06$ & $0.05$ & $0.00$ & $0.00$ \\
\hline
\end{tabular}
\end{table}

A equivalência de desempenho entre \ac{BPSK} e \ac{QPSK} para geração de chaves em camada física é um resultado relevante, indicando que a escolha da modulação pode ser guiada por outros requisitos do sistema (eficiência espectral, simplicidade de implementação) sem impacto significativo na qualidade das chaves geradas. Do ponto de vista de segurança, ambas as modulações fornecem nível equivalente de proteção, uma vez que a segurança do sistema deriva fundamentalmente da descorrelação espacial do canal e não da modulação empregada.

\subsection{Experimento 3: Variação do Código BCH}

O terceiro experimento avalia o desempenho de diferentes códigos \ac{BCH} na etapa de reconciliação: \ac{BCH}(7,4,1), \ac{BCH}(15,7,2) e \ac{BCH}(127,64,10). A Tabela~\ref{tab:exp03_bch} apresenta os resultados obtidos para \ac{SNR} de $11.18$~dB.

\begin{table}[htbp]
\centering
\caption{Comparação de desempenho entre diferentes códigos BCH (SNR = 11.18~dB).}
\label{tab:exp03_bch}
\begin{tabular}{cccccc}
\hline
\textbf{Código} & \textbf{n} & \textbf{k} & \textbf{t} & \textbf{Taxa} & \textbf{KDR (\%)} \\
\hline
BCH(7,4) & 7 & 4 & 1 & 0.57 & 0.77 \\
BCH(15,7) & 15 & 7 & 2 & 0.47 & 0.21 \\
BCH(127,64) & 127 & 64 & 10 & 0.50 & \textbf{0.09} \\
\hline
\end{tabular}
\end{table}

Os resultados confirmam que o código \ac{BCH}(127,64,10), com capacidade de correção de até $t=10$ erros, oferece o melhor compromisso entre robustez e eficiência (taxa de código $R = 0.50$), justificando sua escolha para o sistema proposto. A maior capacidade de correção de erros do código \ac{BCH}(127,64,10) traduz-se diretamente em maior segurança do sistema, pois permite operação confiável (\ac{KDR} $\approx 0\%$) em condições de canal mais adversas, reduzindo a necessidade de retransmissões que poderiam expor informação adicional ao atacante

\subsection{Experimento 4: Análise de Complexidade Computacional}

O quarto experimento mede o tempo de execução das principais operações do sistema para diferentes códigos \ac{BCH}. A Figura~\ref{fig:exp04_complexidade} apresenta graficamente os tempos de codificação e decodificação. Observa-se que o código \ac{BCH}(127,64,10) apresenta tempo total de processamento de 0.489~ms (0.061~ms para codificação e 0.428~ms para decodificação), demonstrando viabilidade para aplicações \ac{IoT} em tempo real que tipicamente exigem latências inferiores a 10~ms. A Tabela~\ref{tab:exp04_complexidade} apresenta os valores numéricos completos.

\begin{figure}[htbp]
\centering
\includegraphics[width=0.85\columnwidth]{figuras/exp04_analise_complexidade_01_20260212_021634.png}
\caption{Complexidade computacional: tempo de codificação versus decodificação para diferentes códigos BCH.}
\label{fig:exp04_complexidade}
\end{figure}

\begin{table}[htbp]
\centering
\caption{Análise de complexidade computacional para diferentes códigos BCH.}
\label{tab:exp04_complexidade}
\begin{tabular}{cccc}
\hline
\textbf{Código} & \textbf{Codificação (ms)} & \textbf{Decodificação (ms)} & \textbf{Total (ms)} \\
\hline
BCH(7,4) & 0.042 & 0.126 & 0.168 \\
BCH(15,7) & 0.061 & 0.163 & 0.224 \\
BCH(127,64) & 0.061 & 0.428 & \textbf{0.489} \\
BCH(255,139) & 0.052 & 0.447 & 0.499 \\
\hline
\end{tabular}
\end{table}

O tempo de processamento de $0.489$~ms para o código \ac{BCH}(127,64) corresponde a uma capacidade teórica de aproximadamente $2044$ operações por segundo, demonstrando a viabilidade do sistema para aplicações de \ac{IoT} em tempo real, mesmo em dispositivos de baixo custo operando em software.

\subsection{Experimento 5: Perfis de Dispositivos IoT}

O quinto experimento valida a aplicabilidade do sistema em cinco perfis representativos de dispositivos \ac{IoT}, cada um caracterizado por parâmetros físicos realistas de mobilidade, frequência de operação e condições de canal. A Figura~\ref{fig:exp05_perfis} apresenta graficamente o comportamento de \ac{BMR} e \ac{KDR} para todos os perfis. A Figura~\ref{fig:exp05_bmr} mostra que o \ac{BMR} inicial varia entre os perfis de acordo com o erro de estimação de canal adotado (8\% para sensor estático, 30\% para veículo urbano). Já a Figura~\ref{fig:exp05_kdr} demonstra que todos os perfis atingem \ac{KDR} inferior a 1\% em 9--11~dB e \ac{KDR} nula em 13~dB, evidenciando robustez do sistema desde sensores estáticos até veículos urbanos a 60~km/h. A Tabela~\ref{tab:exp05_perfis} sumariza os parâmetros e \ac{SNR} mínimo operacional.

\begin{table}[htbp]
\centering
\caption{Perfis de dispositivos IoT e SNR mínimo operacional.}
\label{tab:exp05_perfis}
\begin{tabular}{lcccc}
\hline
\textbf{Perfil} & \textbf{Velocidade} & \textbf{Frequência} & \textbf{$\rho_{\text{temporal}}$} & \textbf{SNR$_{\text{min}}$ (dB)} \\
\hline
Sensor estático & 0~km/h & 868~MHz & 1.000 & 9 \\
Pessoa andando & 5~km/h & 2.4~GHz & 0.940 & 11 \\
Veículo urbano & 60~km/h & 5.9~GHz & 0.160 & 11 \\
Drone & 40~km/h & 2.4~GHz & 0.609 & 11 \\
NB-IoT & 10~km/h & 900~MHz & 0.955 & 11 \\
\hline
\end{tabular}
\end{table}

Observa-se que o perfil de sensor estático apresenta o melhor desempenho, atingindo \ac{KDR}~$<1\%$ em apenas $9$~dB devido à correlação temporal perfeita ($\rho=1.0$) e erro de estimação baixo ($8\%$). Os demais perfis convergem para \ac{KDR}~$<1\%$ em $11$~dB, e todos alcançam \ac{KDR} nula em $13$~dB.

Um resultado notável é a operação bem-sucedida do sistema no cenário de veículo urbano ($60$~km/h, $\rho_{\text{temporal}} = 0.16$), demonstrando que o erro de estimação de canal controlado ($\leq 30\%$) é mais crítico para a viabilidade do sistema do que a correlação temporal propriamente dita. Esse insight sugere que estratégias de estimação robusta de canal são fundamentais para extensão do sistema a cenários de alta mobilidade. Do ponto de vista de segurança, a validação em múltiplos perfis com \ac{SNR} mínimo entre 9--13~dB garante que o sistema opera de forma confiável em condições práticas diversas, mantendo \ac{KDR} nula e, portanto, impedindo que Eve obtenha qualquer informação útil sobre as chaves estabelecidas entre Alice e Bob.

\begin{figure*}[t]
\centering
\subfigure[BMR antes da reconciliação]{\includegraphics[width=0.48\textwidth]{figuras/exp05_perfis_dispositivos_20260212_021230_01.png}\label{fig:exp05_bmr}}
\hfill
\subfigure[KDR após reconciliação BCH(127,64,10)]{\includegraphics[width=0.48\textwidth]{figuras/exp05_perfis_dispositivos_20260212_021230_02.png}\label{fig:exp05_kdr}}
\caption{Desempenho do sistema em cinco perfis representativos de dispositivos IoT.}
\label{fig:exp05_perfis}
\end{figure*}

\subsection{Experimento 6: Análise de Segurança contra Espionagem Passiva}

O sexto experimento investiga a segurança do sistema contra um atacante passivo (Eve) que observa o canal a partir de posições espaciais e temporais distintas. Foram avaliadas duas configurações: descorrelação espacial e descorrelação temporal.

\subsubsection{Descorrelação Espacial}

A Tabela~\ref{tab:exp06_espacial} apresenta a correlação espacial de Eve em relação a Alice e a correspondente \ac{BER} observada por Eve para diferentes distâncias laterais. A frequência de operação foi fixada em $2.4$~GHz ($\lambda = 12.5$~cm) e a \ac{SNR} em $9$~dB (cenário de teste de estresse em condições moderadamente ruidosas).

\begin{table}[htbp]
\centering
\caption{Segurança contra espionagem em função da descorrelação espacial.}
\label{tab:exp06_espacial}
\begin{tabular}{cccc}
\hline
\textbf{Distância Eve (m)} & \textbf{$\lambda/2$} & \textbf{$\rho_{\text{espacial}}$} & \textbf{BER Eve (\%)} \\
\hline
0.10 & 1.6 & 0.210 & 48.46 \\
0.20 & 3.2 & 0.020 & 48.46 \\
0.50 & 8.0 & 0.002 & 48.65 \\
$\geq 1.00$ & $\geq 16$ & $\approx 0$ & $\approx 48.5$ \\
\hline
\end{tabular}
\end{table}

Os resultados demonstram que para distâncias superiores a $0.2$~m (aproximadamente $3.2\lambda/2$), a correlação espacial torna-se desprezível ($\rho < 0.02$), garantindo \ac{BER} de Eve próxima a $50\%$, equivalente a uma tentativa de adivinhação aleatória. Esse comportamento valida experimentalmente o modelo de Clarke~\cite{clarke_1968} e confirma a segurança física do sistema contra espionagem passiva. A Figura~\ref{fig:exp06_eve} ilustra graficamente o decaimento da correlação espacial $\rho(h_{\text{Alice}}, h_{\text{Eve}})$ em função da distância lateral, calculada pelo método de Yuan et al.~\cite{yuan_fast_2013}. Observa-se que a partir de 20~cm ($\approx 3\lambda/2$ a 2.4~GHz), a correlação torna-se desprezível ($\rho < 0.02$), garantindo que Eve não consiga extrair informação útil sobre a chave gerada.

\begin{figure}[htbp]
\centering
\includegraphics[width=0.75\columnwidth]{figuras/exp06_analise_eve_20260212_021230_01.png}
\caption{Descorrelação espacial: correlação de canal entre Alice e Eve versus distância lateral.}
\label{fig:exp06_eve}
\end{figure}

\subsubsection{Descorrelação Temporal}

A análise de descorrelação temporal, realizada mantendo Eve a uma distância fixa de $0.5$~m (já descorrelacionada espacialmente), demonstrou que a correlação temporal é irrelevante para a segurança quando a descorrelação espacial é suficiente. Mesmo com atrasos temporais variando de $0$ a $10$~ms, a correlação total $\rho_{\text{total}} = \rho_{\text{espacial}} \times \rho_{\text{temporal}}$ permaneceu próxima a zero devido ao termo espacial dominante.

\subsection{Experimento 7: Impacto do Guard-Band}

O sétimo e último experimento avalia o impacto da limiarização adaptativa (\textit{guard-band}) sobre a eficiência e segurança do sistema. Variou-se o parâmetro de \textit{guard-band} de $0.0$ (sem zona morta) a $1.0$ (zona morta ampla, em múltiplos de $\sigma_n$), mantendo \ac{SNR} fixa em $15$~dB.

A Tabela~\ref{tab:exp07_guardband} apresenta os resultados obtidos. Observa-se que o sistema apresenta \ac{BER} de Eve próxima a $50\%$ mesmo sem utilização de \textit{guard-band} (GB $= 0$), indicando que o sistema é naturalmente seguro devido à descorrelação espacial. Surpreendentemente, valores elevados de \textit{guard-band} (GB $\geq 0.5$) não apenas reduzem drasticamente a taxa efetiva de geração de chaves (devido ao descarte de bits próximos ao limiar), mas também podem introduzir viés estatístico que degrada ligeiramente a segurança (BER Eve $= 46.82\%$ para GB $= 1.0$).

\begin{table}[htbp]
\centering
\caption{Impacto do guard-band na eficiência e segurança do sistema.}
\label{tab:exp07_guardband}
\begin{tabular}{ccccc}
\hline
\textbf{GB ($\sigma$)} & \textbf{KDR Bob (\%)} & \textbf{BER Eve (\%)} & \textbf{Taxa (kbps)} & \textbf{Descarte (\%)} \\
\hline
0.0 & 0.03 & 49.67 & 127.0 & 0.0 \\
0.1 & 0.00 & 49.92 & 118.4 & 6.8 \\
0.3 & 0.03 & 50.07 & 102.9 & 18.9 \\
0.5 & 0.02 & 49.91 & 89.5 & 29.5 \\
1.0 & 0.03 & 46.82 & 63.1 & 50.3 \\
\hline
\end{tabular}
\end{table}

\begin{figure}[htbp]
\centering
\includegraphics[width=0.85\columnwidth]{figuras/exp07_impacto_guard_band_20260212_021342_02.png}
\caption{BER de Eve versus guard-band (SNR = 15~dB).}
\label{fig:exp07_guardband}
\end{figure}

Esse resultado é particularmente relevante, pois demonstra que técnicas de limiarização adaptativa complexa, frequentemente propostas na literatura, não são necessárias para garantir a segurança do sistema quando a descorrelação espacial é adequadamente explorada. A Figura~\ref{fig:exp07_guardband} confirma que a \ac{BER} de Eve permanece próxima a $50\%$ (chute aleatório) independentemente do guard-band utilizado, mesmo sem limiarização (GB=0). Isso evidencia que a descorrelação espacial é suficiente para garantir a segurança física do sistema sem necessidade de técnicas adaptativas complexas. Recomenda-se operar o sistema com GB $= 0$ ou GB $\leq 0.1\sigma$ para maximizar a taxa de geração de chaves sem comprometer a segurança.

\subsection{Discussão Geral e Comparação com Estado da Arte}

Os resultados experimentais demonstram a viabilidade técnica do sistema proposto para geração de chaves criptográficas em camada física em cenários práticos de \ac{IoT}. O \ac{SNR} mínimo de $13$--$15$~dB identificado no Experimento~1 é plenamente alcançável em aplicações reais, sendo compatível com enlaces de comunicação típicos em redes celulares \ac{5G} e \ac{NB-IoT}~\cite{3gpp_nr_coverage, nb_iot_link_budget}, onde valores superiores a 15~dB são rotineiramente observados em condições normais de operação.

A complexidade computacional de $0.489$~ms (Experimento~4) viabiliza a implementação em dispositivos de baixo custo operando em software, sem necessidade de hardware dedicado (FPGA, USRP), distinguindo este trabalho de abordagens anteriores que requerem plataformas especializadas~\cite{yuan_fast_2013}. Essa característica é fundamental para massificação do sistema em redes \ac{IoT}, onde bilhões de dispositivos de baixo custo precisam estabelecer chaves de forma eficiente e segura.

A validação em cinco perfis de dispositivos \ac{IoT} (Experimento~5) demonstra a ampla aplicabilidade do sistema, cobrindo desde sensores estáticos até veículos urbanos a $60$~km/h. O funcionamento em cenários de alta mobilidade ($\rho_{\text{temporal}} = 0.16$) sugere que o erro de estimação de canal, quando controlado adequadamente, é mais crítico que a correlação temporal, um insight relevante para projeto de sistemas práticos que contradiz pressupostos comuns na literatura que enfatizam excessivamente a necessidade de alta correlação temporal~\cite{mathur_pks}.

A análise de segurança (Experimento~6) confirma a proteção contra espionagem passiva a partir de distâncias de $20$~cm, validando experimentalmente o modelo teórico de Clarke~\cite{clarke_1968}. A \ac{BER} de Eve próxima a 50\% representa segurança informacional perfeita no sentido teórico da informação~\cite{maurer_secret_key}, uma vez que Eve não obtém nenhuma informação útil sobre a chave estabelecida, mesmo com capacidade computacional ilimitada. Esse nível de segurança contrasta favoravelmente com sistemas criptográficos tradicionais, cuja proteção baseia-se em hipóteses de complexidade computacional vulneráveis a ataques quânticos.

Finalmente, a demonstração de que \textit{guard-band} não é necessário (Experimento~7) representa uma contribuição original, simplificando a implementação prática do sistema sem comprometer a segurança. Esse resultado desafia trabalhos anteriores que propõem mecanismos complexos de limiarização adaptativa~\cite{zeng_pkg_challenges}, demonstrando que em sistemas baseados em correlação espacial (ao invés de reciprocidade temporal), a descorrelação natural do canal é suficiente para garantir segurança física robusta.
%\section{Conclusões}\label{sec:conclusao}


% Seção 08 - Seção VII

\printbibliography


\end{document}
