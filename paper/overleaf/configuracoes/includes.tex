%Use esse arquivo para incluir novos pacotes

\usepackage[%usado para determinar medidas
top=1.78cm,
bottom=1.78cm,
left=1.65cm,
right=1.65cm,
headsep=0cm,
%showframe
]{geometry}
%\usepackage[justification=centering]{caption}
\usepackage{times}
\usepackage{enumitem}%redefinir espacos itemize
\usepackage{graphicx}
\usepackage{url,hyperref}
\usepackage[utf8]{inputenc}
\usepackage{float}%mais controle para manipular figuras
\usepackage{caption}%manipular legenda da figura e tabela
\usepackage{mathtools}%equacoes
\usepackage[hang,flushmargin]{footmisc} 
\usepackage{xcolor}
\usepackage{wrapfig} %usado para envolver figura com texto
%\usepackage[portuguese]{babel}
\usepackage{fancyhdr}%criacao do cabecalho
\usepackage{etoolbox}
\usepackage[export]{adjustbox}%mais controle para ajustar tamanho da tabela
\usepackage{multirow}%usado para mesclar linhas em tabelas
\usepackage{comment}%ambiente para comentario
\usepackage{relsize} %usado por comandos \mathlarger
\usepackage{subfigure}
\usepackage{acro}


%Referencia bibliografica
\usepackage[
    style=numeric,
    sorting=none,
    maxbibnames=10]{biblatex}
\addbibresource{referencia.bib}

%Idioma. Use "english" para trabalhos em inglês
\usepackage[brazil]{babel}

%Inserindo Códigos
\usepackage{listings}
\definecolor{dkgreen}{rgb}{0,0.6,0}
\definecolor{gray}{rgb}{0.5,0.5,0.5}
\definecolor{mauve}{rgb}{0.58,0,0.82}
%\lstset{
%  language=Python,                
%  basicstyle=\footnotesize,           
%   numbers=false,                   
%   numberstyle=\tiny\color{gray},  
%   stepnumber=2,                             
%   numbersep=5pt,                  
%   backgroundcolor=\color{white},    
%   showspaces=false,               
%   showstringspaces=false,         
%   showtabs=false,                 
%   frame=single,                   
%   rulecolor=\color{black},        
%   tabsize=2,                      
%   captionpos=b,                   
%   breaklines=true,                
%   breakatwhitespace=false,        
%   title=\lstname,                               
%   keywordstyle=\color{blue},          
%   commentstyle=\color{dkgreen},       
%   stringstyle=\color{mauve}    
% }